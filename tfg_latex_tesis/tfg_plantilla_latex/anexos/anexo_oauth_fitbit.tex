\chapter{Anexo: Flujo de Autenticación OAuth 2.0 con Fitbit}
\label{anexo:oauth_fitbit}

Este anexo describe en detalle la implementación del flujo \textit{Authorization Code Grant} de OAuth 2.0 utilizado para la integración con Fitbit\textsuperscript{\textregistered}. Como se mencionó en la Sección \ref{subsec:ea_fitbit_api_oauth}, este flujo es el más adecuado para aplicaciones web con backend, ya que permite una gestión segura de las credenciales y tokens \cite{oauth2_spec}.

\section{Visión General del Flujo de Autenticación}

El flujo de autenticación implementado consta de los siguientes pasos:

\begin{enumerate}
    \item \textbf{Inicio de Sesión:} El personal autorizado accede a la aplicación web mediante sus credenciales verificadas.
    
    \item \textbf{Selección de Usuario:} Se selecciona al residente cuya cuenta de Fitbit\textsuperscript{\textregistered} se desea vincular. El residente debe estar previamente registrado en el sistema con:
    \begin{itemize}
        \item Nombre completo.
        \item Correo electrónico asociado a su cuenta de Fitbit\textsuperscript{\textregistered}.
        \item Consentimiento explícito documentado para la recolección de datos.
    \end{itemize}
    
    \item \textbf{Construcción de la URL de Autorización:} La aplicación genera una URL de autorización con los siguientes parámetros:
    \begin{itemize}
        \item \texttt{client\_id}: Identificador único de la aplicación proporcionado por Fitbit.
        \item \texttt{redirect\_uri}: URI de redirección registrada en la aplicación de Fitbit.
        \item \texttt{response\_type}: Valor fijo \texttt{code}, indicando que se solicita un código de autorización.
        \item \texttt{scope}: Lista de permisos solicitados, separados por espacios (ver Tabla \ref{tab:fitbit_scopes}).
        \item \texttt{state}: Valor aleatorio generado para prevenir ataques CSRF.
    \end{itemize}
    
    \item \textbf{Redirección al Servidor de Autorización:} El navegador del usuario es redirigido a la URL de autorización de Fitbit, donde se solicita al residente que inicie sesión y otorgue los permisos solicitados.
    
    \item \textbf{Obtención del Código de Autorización:} Tras la autorización, Fitbit redirige al navegador a la \texttt{redirect\_uri} especificada, incluyendo los siguientes parámetros:
    \begin{itemize}
        \item \texttt{code}: Código de autorización temporal, válido por 10 minutos.
        \item \texttt{state}: Valor original para verificación.
    \end{itemize}
    
    \item \textbf{Intercambio del Código por Tokens:} El backend de la aplicación realiza una solicitud POST al endpoint \texttt{https://api.fitbit.com/oauth2/token} con los siguientes datos:
    \begin{itemize}
        \item \texttt{client\_id} y \texttt{client\_secret}: Credenciales de la aplicación.
        \item \texttt{code}: Código de autorización recibido.
        \item \texttt{grant\_type}: Valor fijo \texttt{authorization\_code}.
        \item \texttt{redirect\_uri}: Debe coincidir con la URI utilizada anteriormente.
    \end{itemize}
    La autenticación se realiza mediante encabezados HTTP con codificación \texttt{Basic} de las credenciales.
    
    \item \textbf{Almacenamiento Seguro de Tokens:} Los tokens de acceso y de refresco recibidos son almacenados de forma segura en la base de datos, asociados al ID del residente y cifrados para proteger su confidencialidad.
    
    \item \textbf{Renovación Automática de Tokens:} El sistema implementa un mecanismo para renovar automáticamente los tokens de acceso antes de su expiración, utilizando el token de refresco correspondiente.
\end{enumerate}

\section{Scopes de Acceso Solicitados}

Los scopes determinan los tipos de datos a los que la aplicación puede acceder. A continuación se presenta una tabla con los scopes utilizados y su descripción:


\begin{table}[H]
    \centering
    \caption{Scopes de acceso solicitados a Fitbit\textsuperscript{\textregistered}}
    \label{tab:fitbit_scopes}
    \begin{tabular}{|l|p{10cm}|}
    \hline
    \textbf{Scope} & \textbf{Descripción} \\
    \hline
    \texttt{activity} & Acceso a datos de actividad física, como pasos, distancia y calorías quemadas. \\
    \hline
    \texttt{cardio\_fitness} & Acceso a métricas de condición cardiovascular como VO2 max. \\
    \hline
    \texttt{electrocardiogram} & Acceso a lecturas de electrocardiograma cuando estén disponibles. \\
    \hline
    \texttt{heartrate} & Acceso a datos de frecuencia cardíaca. \\
    \hline
    \texttt{irregular\_rhythm\_notifications} & Acceso a notificaciones de ritmos cardíacos irregulares. \\
    \hline
    \texttt{location} & Acceso a datos de ubicación durante actividades. \\
    \hline
    \texttt{nutrition} & Acceso a registros de alimentación y nutrición. \\
    \hline
    \texttt{oxygen\_saturation} & Acceso a lecturas de saturación de oxígeno en sangre. \\
    \hline
    \texttt{profile} & Acceso a información básica del perfil del usuario. \\
    \hline
    \texttt{respiratory\_rate} & Acceso a datos de frecuencia respiratoria. \\
    \hline
    \texttt{settings} & Acceso a configuraciones del dispositivo. \\
    \hline
    \texttt{sleep} & Acceso a datos de sueño, incluyendo duración y etapas. \\
    \hline
    \texttt{social} & Acceso a datos sociales como amigos y grupos. \\
    \hline
    \texttt{temperature} & Acceso a datos de temperatura corporal y de la piel. \\
    \hline
    \texttt{weight} & Acceso a mediciones de peso y composición corporal. \\
    \hline
    \end{tabular}
    \end{table}

Es importante seleccionar únicamente los scopes necesarios para minimizar la exposición de datos sensibles y cumplir con el principio de minimización de datos del RGPD.

\section{Consideraciones de Seguridad}

Para garantizar la seguridad del flujo de autenticación, se han implementado las siguientes medidas:

\begin{itemize}
    \item \textbf{Prevención de CSRF:} Se utiliza un parámetro \texttt{state} aleatorio en la URL de autorización, que se verifica al recibir la respuesta para prevenir ataques de tipo Cross-Site Request Forgery.
    
    \item \textbf{Almacenamiento Seguro de Tokens:} Los tokens se almacenan cifrados en la base de datos, y el acceso a ellos está restringido a componentes autorizados del sistema.
    
    \item \textbf{Gestión de Tokens de Refresco:} Se implementa un mecanismo para renovar los tokens de acceso utilizando los tokens de refresco antes de su expiración, garantizando un acceso continuo sin requerir reautenticación frecuente.
    
    \item \textbf{Cumplimiento del RGPD:} Se asegura que el procesamiento de datos personales cumpla con los principios del Reglamento General de Protección de Datos, incluyendo la obtención de consentimiento informado y la implementación de medidas técnicas y organizativas adecuadas.
\end{itemize}

\section{Construcción de la URL de Autorización}

La URL de autorización se construye concatenando los parámetros mencionados anteriormente. Un ejemplo de URL sería:

\begin{verbatim}
https://www.fitbit.com/oauth2/authorize?
response_type=code&
client_id=YOUR_CLIENT_ID&
redirect_uri=YOUR_REDIRECT_URI&
scope=activity heartrate sleep profile&
state=RANDOM_STATE_STRING
\end{verbatim}

Es fundamental que la \texttt{redirect\_uri} coincida exactamente con la registrada en la configuración de la aplicación en el portal de desarrolladores de Fitbit.


\section{Referencias}

\begin{itemize}
    \item Fitbit Web API Documentation \cite{fitbit_api_docs}
    \item OAuth 2.0 Authorization Framework \cite{Auth_Tutorial}
    \item RGPD \cite{gdpr2016}
\end{itemize}
