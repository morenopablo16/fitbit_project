\chapter{Resultados Detallados de las Pruebas}
\label{anexo:pruebas}

Este anexo presenta los resultados completos de los tests automatizados desarrollados para validar el sistema de alertas, incluyendo logs, configuraciones y comportamiento observado.

\section{Entorno de Pruebas}
\label{anexo:pruebas:entorno}

Las pruebas se ejecutaron en entorno local controlado:
\begin{itemize}
    \item CPU: Intel Core i7-9750H (6 cores, 12 threads)
    \item RAM: 16GB DDR4
    \item Sistema Operativo: Ubuntu 20.04 LTS
    \item Base de datos: PostgreSQL 12.4 + TimescaleDB
    \item Lenguaje: Python 3.8.5, Flask 2.0.1
\end{itemize}

\section{Datos Insertados}
\label{anexo:pruebas:insertados}

Se generaron datos sintéticos para 3 usuarios durante 23 días:

\begin{itemize}
    \item 20 días de datos normales.
    \item 3 días de datos anómalos específicos.
    \item Datos intradía: frecuencia cardíaca y pasos.
    \item Datos faltantes y fisiológicamente imposibles.
\end{itemize}

\section{Resultados del Test de Umbrales}
\label{anexo:pruebas:umbrales}

Para cada tipo de alerta, se esperaban valores concretos. Se resume a continuación el comportamiento real comparado con lo esperado:

\begin{verbatim}
User ID: 2
✓ activity_drop
✓ sedentary_increase
✗ sleep_duration_change (Expected: True, Triggered: False)
✓ heart_rate_anomaly
✓ data_quality
✓ intraday_activity_drop
\end{verbatim}

\noindent Resultado general del test:

\begin{itemize}
    \item Alertas totales evaluadas: 36
    \item Esperadas y disparadas: 18
    \item Falsos positivos: 0
    \item Falsos negativos: 3 (solo en sueño)
    \item Precisión: 100.0\%
    \item Recall: 83.3\%
\end{itemize}

\section{Validación por Combinación de Anomalías}
\label{anexo:pruebas:combinadas}

En un día específico, se activaron simultáneamente:

\begin{itemize}
    \item Sedentarismo extremo (900 min)
    \item HR anómala (bpm = 60, std = 4.0)
    \item Calidad de datos (oxígeno = 18.5\%)
\end{itemize}

\noindent Resultado: se activaron todas las alertas previstas con sus prioridades correspondientes.

\section{Ejemplo de Logs: Test de Alertas Combinadas}
\label{anexo:pruebas:logs_combinadas}

A continuación se muestra un extracto real del archivo \texttt{summary\_combined.log} generado por el test automatizado de combinación de anomalías clínicas. Este log resume, para cada usuario y fecha, las alertas esperadas, las alertas realmente generadas por el sistema y un análisis de coincidencias o diferencias:

\begin{verbatim}
=== RESUMEN DE PRUEBAS DE ALERTAS COMBINADAS ===

Usuario 1:

Fecha: 2024-04-15 00:00:00
Alertas esperadas:
  - activity_drop
  - sedentary_increase
  - data_quality

Alertas generadas:
  - activity_drop
  - sleep_duration_change
  - sedentary_increase
  - data_quality

Análisis:
  ⚠ Diferencias encontradas:
    - Alertas adicionales: sleep_duration_change
    Nota: Las alertas adicionales son correctas según los umbrales definidos

==================================================

Fecha: 2024-04-16 00:00:00
Alertas esperadas:
  - activity_drop
  - sedentary_increase

Alertas generadas:
  - activity_drop
  - data_quality

Análisis:
  ✓ Todas las alertas esperadas fueron generadas correctamente

==================================================
\end{verbatim}

\textbf{Interpretación:}
\begin{itemize}
    \item En el primer escenario (15/04/2024), el sistema generó todas las alertas esperadas y, además, una alerta de \texttt{sleep\_duration\_change}. Esta alerta adicional es coherente, ya que la reducción del sueño superó el umbral configurado para disparar la alerta, aunque no se había previsto explícitamente en el test.
    \item En el segundo escenario (16/04/2024), todas las alertas esperadas fueron generadas correctamente, y la aparición de la alerta de calidad de datos (\texttt{data\_quality}) es aceptable según los criterios implementados.
\end{itemize}

Este patrón se repite para los tres usuarios de prueba, confirmando que el sistema es capaz de detectar y reportar múltiples anomalías simultáneamente, sin interferencias entre reglas y con sensibilidad adecuada a los umbrales clínicos definidos.

\vspace{1em}
\noindent\textbf{Nota:} Los logs completos, incluyendo los detalles de cada test y los datos sintéticos utilizados, se encuentran disponibles en el repositorio del proyecto bajo \texttt{tests/logs/combined\_alerts/}.

\section{Observaciones Relevantes}
\label{anexo:pruebas:observaciones}

- Las alertas se generaron independientemente.
- No hubo colisiones ni interferencias.
- El sistema maneja datos incompletos de forma fiable.

\section{Logs y Archivos}
\label{anexo:pruebas:logs}

Los logs generados por los tests (\texttt{summary.log} y \texttt{detailed.log}) se encuentran disponibles en el repositorio del proyecto.