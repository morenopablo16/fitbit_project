\chapter{Resultados Detallados de las Pruebas}
\label{anexo:pruebas}

Este anexo presenta los resultados completos de los tests automatizados desarrollados para validar el sistema de alertas, incluyendo logs, configuraciones y comportamiento observado.

\section{Entorno de Pruebas}
\label{anexo:pruebas:entorno}

Las pruebas se ejecutaron en entorno local controlado:
\begin{itemize}
    \item CPU: Intel Core i7-9750H (6 cores, 12 threads)
    \item RAM: 16GB DDR4
    \item Sistema Operativo: Ubuntu 20.04 LTS
    \item Base de datos: PostgreSQL 12.4 + TimescaleDB
    \item Lenguaje: Python 3.8.5, Flask 2.0.1
\end{itemize}

\section{Datos Insertados}
\label{anexo:pruebas:insertados}

Se generaron datos sintéticos para 3 usuarios durante 23 días:

\begin{itemize}
    \item 20 días de datos normales.
    \item 3 días de datos anómalos específicos.
    \item Datos intradía: frecuencia cardíaca y pasos.
    \item Datos faltantes y fisiológicamente imposibles.
\end{itemize}

\section{Resultados del Test de Umbrales}
\label{anexo:pruebas:umbrales}

Para cada tipo de alerta, se esperaban valores concretos. Se resume a continuación el comportamiento real comparado con lo esperado:

\begin{verbatim}
User ID: 2
✓ activity_drop
✓ sedentary_increase
✗ sleep_duration_change (Expected: True, Triggered: False)
✓ heart_rate_anomaly
✓ data_quality
✓ intraday_activity_drop
\end{verbatim}

\noindent Resultado general del test:

\begin{itemize}
    \item Alertas totales evaluadas: 36
    \item Esperadas y disparadas: 18
    \item Falsos positivos: 0
    \item Falsos negativos: 3 (solo en sueño)
    \item Precisión: 100.0\%
    \item Recall: 83.3\%
\end{itemize}

\section{Validación por Combinación de Anomalías}
\label{anexo:pruebas:combinadas}

En un día específico, se activaron simultáneamente:

\begin{itemize}
    \item Sedentarismo extremo (900 min)
    \item HR anómala (bpm = 60, std = 4.0)
    \item Calidad de datos (oxígeno = 18.5\%)
\end{itemize}

\noindent Resultado: se activaron todas las alertas previstas con sus prioridades correspondientes.

\section{Observaciones Relevantes}
\label{anexo:pruebas:observaciones}

- Las alertas se generaron independientemente.
- No hubo colisiones ni interferencias.
- El sistema maneja datos incompletos de forma fiable.

\section{Logs y Archivos}
\label{anexo:pruebas:logs}

Los logs generados por los tests (\texttt{summary.log} y \texttt{detailed.log}) se encuentran disponibles en el repositorio del proyecto.