\begin{table}[htbp]
    \centering
    \renewcommand{\arraystretch}{1.25}
    \begin{tabularx}{\textwidth}{|l|c|c|X|}
        \hline
        \textbf{Tipo de Alerta} & \textbf{Prioridad} & \textbf{Umbral / Rango} & \textbf{Justificación y Referencia} \\
        \hline
        Caída en actividad física & Alta & $>$50\% reducción & Asociado a deterioro funcional acelerado y mayor riesgo de hospitalización en mayores. \newline \textit{Smith et al., 2019} \\
        Caída en actividad física & Media & $>$30\% reducción & Cambio significativo en el patrón habitual, permite intervención temprana. \newline \textit{Asociación Americana de Geriatría} \\
        \hline
        Aumento de tiempo sedentario & Alta & $>$50\% incremento & Relación dosis-respuesta con riesgo cardiovascular y metabólico. \newline \textit{Owen et al., 2020} \\
        Aumento de tiempo sedentario & Media & $>$30\% incremento & Predice mayor riesgo de hospitalización y deterioro funcional. \newline \textit{Estudio LIFE} \\
        \hline
        Cambio en duración del sueño & Alta & $>$30\% variación (aumento o disminución) & Cambios de $\pm$30\% (2-2.5h) asociados a trastornos neurológicos y psiquiátricos. \newline \textit{Irwin, 2015} \\
        \hline
        Anomalía en frecuencia cardíaca & Alta & $>$2 desviaciones estándar, $>$20\% lecturas anómalas & Patrón sostenido de irregularidad, riesgo de eventos cardiovasculares. \newline \textit{Chow et al., 2018} \\
        Anomalía en frecuencia cardíaca & Media & $>$2 desviaciones estándar, $>$10\% lecturas anómalas & Detecta anomalías relevantes evitando falsas alarmas. \\
        \hline
        Validación de datos & - & Rango fisiológico: \newline Pasos: 0-50,000 \newline FC: 30-200 bpm \newline Sueño: 0-1440 min \newline Sedentarismo: 0-1440 min \newline SpO$_2$: 80-100\% & Basado en límites fisiológicos y clínicos. Valores fuera de rango indican error de medición o situación crítica. \\
        \hline
        Inactividad intradía & Media/Alta & $\geq$2h sin pasos & Períodos prolongados de inactividad aumentan riesgo cardiovascular y de caídas. \newline \textit{Barone Gibbs et al., 2021; Owen et al., 2020} \\
        \hline
    \end{tabularx}
    \caption{Umbrales y criterios de alerta implementados en el sistema, con justificación clínica y referencias.}
    \label{tab:anexo_umbrales_alertas}
\end{table}
