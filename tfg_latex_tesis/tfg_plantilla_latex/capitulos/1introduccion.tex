% -*- coding: utf-8 -*- % Para asegurar codificación correcta
\chapter{Introducción}
\label{chap:introduccion}
A continuación se introduce el contexto y la motivación de este proyecto, para luego definir sus objetivos, alcance y la organización general de este documento
\section{Contexto y Motivación}
\label{sec:intro_contexto}

El envejecimiento de la población es una realidad demográfica global, especialmente acentuada en España, donde, según el Instituto Nacional de Estadística, más del 20\% de la población tiene más de 65 años y se prevé que esta cifra siga aumentando en las próximas décadas \cite{ine_proyeccion_2022_2072}. Este fenómeno implica un incremento en la prevalencia de enfermedades crónicas y una mayor demanda de servicios sanitarios y de cuidados de larga duración, lo que genera una presión significativa sobre los sistemas de salud y las familias \cite{who_ageing_health}. En este escenario, la tecnología emerge como un aliado fundamental para desarrollar soluciones innovadoras que mejoren la calidad de vida de las personas mayores, promuevan su autonomía y faciliten su cuidado.

La monitorización remota de la salud, apoyada en dispositivos electrónicos y sensores, ofrece un enorme potencial para el seguimiento continuo y no invasivo de indicadores fisiológicos y de actividad \cite{majumder2017wearable}. Los dispositivos vestibles o \textit{wearables}, como las pulseras de actividad y los relojes inteligentes, se han popularizado enormemente en los últimos años. Entre ellos, Fitbit\textsuperscript{\textregistered} destaca por su amplia adopción, facilidad de integración mediante API y coste accesible, lo que lo convierte en una opción idónea para proyectos de monitorización remota.

La motivación principal de este Trabajo Fin de Grado (TFG) radica en explorar y aprovechar el potencial de estos dispositivos comerciales, específicamente los de Fitbit\textsuperscript{\textregistered}, para construir un sistema que facilite la monitorización remota de personas mayores por parte de sus cuidadores o personal sanitario. El objetivo es contribuir a una atención más proactiva, permitiendo detectar posibles cambios en el estado de salud de forma temprana y mejorando la tranquilidad de los familiares y responsables del cuidado, respetando la autonomía y privacidad del usuario monitorizado.

\section{Definición del Problema}
\label{sec:intro_problema}

A pesar de la disponibilidad de datos generados por dispositivos como Fitbit\textsuperscript{\textregistered}, a menudo existe una brecha entre la simple recolección de datos y su utilización efectiva para el seguimiento de la salud, especialmente en el contexto de personas mayores y sus cuidadores. Los problemas específicos que este trabajo busca abordar son:

\begin{itemize}
    \item La falta de sistemas integrados que recopilen automáticamente datos relevantes de Fitbit\textsuperscript{\textregistered} y los presenten de forma clara, contextualizada y comprensible para cuidadores no necesariamente expertos en tecnología.
    \item La dificultad para realizar un seguimiento longitudinal de los indicadores clave de salud obtenidos a través de estos dispositivos, identificando tendencias o desviaciones significativas de los patrones habituales del usuario.
    \item La necesidad de implementar soluciones técnicas robustas que gestionen de forma segura la autenticación con servicios de terceros (Fitbit API) y que respeten escrupulosamente la privacidad y la normativa de protección de datos (como el RGPD) al manejar información personal y sensible de salud.
    \item La ausencia frecuente de arquitecturas flexibles y escalables en prototipos de este tipo, que permitan una futura expansión para incluir más usuarios, más tipos de datos o integración con otros sistemas.
\end{itemize}

Este TFG se enfoca en el diseño e implementación de un prototipo que dé respuesta a estos desafíos, proporcionando una solución técnica robusta, funcional y bien documentada.

\section{Objetivos}
\label{sec:intro_objetivos}

Para abordar el problema definido, se establecen los siguientes objetivos, diferenciando entre el objetivo general y los específicos:

\subsection{Objetivo General}
\label{subsec:obj_general}

Diseñar e implementar un prototipo de sistema software para la monitorización remota de indicadores de salud de personas mayores, utilizando datos obtenidos de pulseras de actividad Fitbit\textsuperscript{\textregistered} y presentando la información de forma útil y accesible para cuidadores o personal autorizado, con un enfoque en la seguridad, la privacidad y la escalabilidad.

\subsection{Objetivos Específicos}
\label{subsec:obj_especificos}

\begin{enumerate}
    \item Investigar en profundidad la API web de Fitbit\textsuperscript{\textregistered}, su modelo de datos, las políticas de acceso, la granularidad de los datos disponibles y el proceso de autorización seguro mediante OAuth 2.0.
    \item Diseñar una arquitectura software para el backend del sistema, basada en microservicios ligeros implementados en Python (Flask), que sea modular y escalable.
    \item Implementar un microservicio responsable de gestionar la autenticación de los usuarios del sistema (cuidadores, administradores) y la obtención segura de tokens de acceso para la API de Fitbit\textsuperscript{\textregistered} mediante el flujo OAuth 2.0.
    \item Desarrollar un microservicio encargado de la adquisición periódica y automatizada de los datos de interés de los usuarios desde la API de Fitbit\textsuperscript{\textregistered} (frecuencia cardíaca diaria/intradía, resumen del sueño, pasos y niveles de actividad).
    \item Seleccionar, configurar e implementar una base de datos relacional PostgreSQL con extensión TimescaleDB, adecuada para el almacenamiento eficiente y la consulta de los datos biométricos y de actividad a lo largo del tiempo.
    \item Implementar la lógica necesaria para el procesamiento básico de los datos adquiridos, incluyendo la validación, limpieza y transformación de los datos para su almacenamiento y posterior visualización (cálculo de promedios, identificación de periodos de inactividad/actividad).
    \item Desarrollar un panel de visualización web (dashboard) sencillo e intuitivo utilizando Flask y plantillas HTML, que permita a los cuidadores consultar el histórico de datos mediante gráficos interactivos y visualizar indicadores clave.
    \item Definir e implementar un conjunto inicial de criterios basados en evidencia para la detección temprana de posibles anomalías en los datos de actividad, sueño y frecuencia cardíaca, generando alertas contextualizadas para el personal autorizado.
    \item Analizar y aplicar las medidas técnicas y organizativas necesarias para garantizar la seguridad (autenticación, autorización, protección contra ataques comunes) y privacidad de los datos (minimización, seudonimización si aplica, gestión de consentimientos), en cumplimiento con los principios del Reglamento General de Protección de Datos (RGPD).
    \item Validar el correcto funcionamiento del prototipo desarrollado mediante un conjunto definido de pruebas funcionales (cobertura de requisitos), de integración (comunicación entre microservicios y con API externa) y del sistema (flujos de usuario principales).
\end{enumerate}

\section{Alcance y Limitaciones}
\label{sec:intro_alcance}

El sistema desarrollado en este TFG es un prototipo funcional que demuestra la viabilidad técnica de la solución propuesta. El alcance del trabajo cubre:

\begin{itemize}
    \item Integración con la API de Fitbit\textsuperscript{\textregistered} para obtener datos de frecuencia cardíaca (resumen diario y/o intradía según disponibilidad de la API), patrones de sueño (fases, duración) y actividad física (pasos, minutos activos).
    \item Desarrollo de un backend modular basado en microservicios ligeros implementados en Python (Flask).
    \item Almacenamiento de los datos en una base de datos relacional PostgreSQL con TimescaleDB.
    \item Implementación completa del flujo de autenticación OAuth 2.0 con Fitbit\textsuperscript{\textregistered} para la autorización segura por parte del usuario.
    \item Desarrollo de una interfaz web básica para cuidadores, que permite visualizar gráficos históricos de los datos y gestionar usuarios monitorizados.
    \item Implementación de lógica para evaluar criterios de alerta predefinidos sobre los datos de actividad, sueño y frecuencia cardíaca.
    \item Aplicación de principios de diseño orientados a la seguridad y al cumplimiento del RGPD (por ejemplo, cifrado de datos sensibles en reposo y en tránsito, gestión segura de tokens).
\end{itemize}

Es importante destacar las siguientes limitaciones:

\begin{itemize}
    \item El sistema \textbf{no es un dispositivo médico certificado} y la información proporcionada no debe utilizarse para autodiagnóstico ni para sustituir la consulta con un profesional sanitario cualificado. Su propósito es informativo y de apoyo al cuidado.
    \item La funcionalidad está limitada a los datos y la granularidad que Fitbit\textsuperscript{\textregistered} expone a través de su API web estándar (por ejemplo, la frecuencia cardíaca intradía puede estar limitada a intervalos de 1 minuto o más, dependiendo del nivel de acceso a la API).
    \item El sistema de alertas implementado se basa en criterios iniciales y no incluye mecanismos avanzados de inteligencia artificial para predicción o adaptación dinámica de umbrales. La validación clínica de las alertas generadas queda fuera del alcance de este prototipo.
    \item El prototipo ha sido validado funcionalmente en un entorno de desarrollo y pruebas, pero no ha sido sometido a pruebas de carga extensivas ni a una evaluación de usabilidad formal con usuarios finales (personas mayores y cuidadores).
    \item La interfaz de usuario, aunque funcional, representa un diseño básico y podría beneficiarse de mejoras significativas en términos de experiencia de usuario (UX) y diseño de interfaz (UI) para una adopción real.
    \item El análisis de cumplimiento del RGPD se basa en los principios de diseño y buenas prácticas, pero no constituye una auditoría legal completa ni garantiza la conformidad total en un entorno de producción sin revisiones adicionales.
    \item El sistema no incluye mecanismos avanzados de detección de anomalías o predicción basados en inteligencia artificial, aunque la arquitectura sentaría las bases para su futura integración.
\end{itemize}

\section{Estructura del Documento}
\label{sec:intro_estructura}

La memoria se organiza en los siguientes capítulos:

\begin{itemize}
    \item \textbf{Capítulo 1: Introducción.} (Este capítulo) Presenta el contexto, la motivación, el problema a resolver, los objetivos, el alcance y la estructura del documento.
    \item \textbf{Capítulo 2: Estado del Arte y Marco Tecnológico.} Revisa soluciones existentes en el ámbito de la monitorización remota de salud con wearables y describe en detalle las tecnologías clave seleccionadas y empleadas en el proyecto (Fitbit API, OAuth 2.0, microservicios, bases de datos de series temporales, etc.).
    \item \textbf{Capítulo 3: Análisis y Metodología.} Detalla los requisitos funcionales (lo que el sistema debe hacer) y no funcionales (atributos de calidad como rendimiento, seguridad, usabilidad) identificados para el sistema, y describe brevemente la metodología de desarrollo seguida (ej. iterativa, basada en prototipos).
    \item \textbf{Capítulo 4: Diseño y Arquitectura del Sistema.} Expone las decisiones de diseño tomadas, presentando la arquitectura general del sistema, el diseño detallado de los microservicios del backend, el esquema de la base de datos, el flujo de datos y la integración con la API externa.
    \item \textbf{Capítulo 5: Implementación.} Describe los detalles concretos de la implementación de los componentes más relevantes del sistema, incluyendo el entorno de desarrollo, las librerías principales utilizadas, fragmentos de código ilustrativos y los desafíos técnicos encontrados y cómo fueron resueltos.
    \item \textbf{Capítulo 6: Pruebas y Validación.} Explica la estrategia de pruebas definida y llevada a cabo (pruebas unitarias, de integración, del sistema) para asegurar la calidad del software y validar que el prototipo cumple con los requisitos especificados.
    \item \textbf{Capítulo 7: Resultados y Discusión.} Presenta el prototipo funcional resultante, mostrando ejemplos de su operación (ej. capturas de pantalla del dashboard) y discute los resultados obtenidos en términos de cumplimiento de objetivos, rendimiento observado y las limitaciones inherentes al sistema desarrollado.
    \item \textbf{Capítulo 8: Conclusiones y Trabajo Futuro.} Resume las principales conclusiones extraídas del desarrollo del TFG, destacando las contribuciones del trabajo y proponiendo posibles líneas de mejora, expansión y trabajo futuro sobre el sistema desarrollado.
\end{itemize}