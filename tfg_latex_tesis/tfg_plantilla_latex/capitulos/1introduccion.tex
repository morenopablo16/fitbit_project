% -*- coding: utf-8 -*- % Para asegurar codificación correcta
\chapter{Introducción}
\label{chap:introduccion}
A continuación se introduce el contexto y la motivación de este proyecto, para luego definir sus objetivos, alcance y la organización general de este documento.

\section{Contexto y Motivación}
\label{sec:intro_contexto}

El envejecimiento de la población es una realidad demográfica global, especialmente acentuada en España, donde, según el Instituto Nacional de Estadística, más del 20\% de la población tiene más de 65 años y se prevé que esta cifra siga aumentando en las próximas décadas \cite{ine_proyeccion_2022_2072}. Este fenómeno implica un incremento en la prevalencia de enfermedades crónicas y una mayor demanda de servicios sanitarios y de cuidados de larga duración, lo que genera una presión significativa sobre los sistemas de salud y las familias \cite{who_ageing_health}.

La monitorización remota de la salud, apoyada en dispositivos vestibles como Fitbit\textsuperscript{\textregistered}, ofrece un potencial significativo para el seguimiento continuo y no invasivo de indicadores fisiológicos y de actividad \cite{majumder2017wearable}. Estos dispositivos, que se han popularizado por su facilidad de uso y coste accesible, generan datos valiosos que, adecuadamente procesados, pueden contribuir a una atención más proactiva y personalizada.

La motivación principal de este Trabajo Fin de Grado (TFG) es desarrollar un sistema que aproveche los datos de dispositivos Fitbit\textsuperscript{\textregistered} para facilitar la monitorización remota de personas mayores, permitiendo detectar cambios significativos en su estado de salud de forma temprana y mejorando la tranquilidad de cuidadores y familiares.

\section{Definición del Problema}
\label{sec:intro_problema}

A pesar de la disponibilidad de datos generados por dispositivos como Fitbit\textsuperscript{\textregistered}, existen dos desafíos principales que este trabajo busca abordar:

\begin{itemize}
    \item La necesidad de un sistema integrado que recopile automáticamente datos de Fitbit\textsuperscript{\textregistered} y los presente de forma clara y contextualizada para cuidadores, permitiendo un seguimiento longitudinal efectivo de indicadores clave de salud.
    \item La importancia de implementar una solución técnica robusta que gestione de forma segura la autenticación, respete la privacidad de los datos de salud según el RGPD, y permita una futura expansión del sistema.
    \item La falta de sistemas integrados que recopilen automáticamente datos relevantes de Fitbit\textsuperscript{\textregistered} y los presenten de forma clara, contextualizada y comprensible para cuidadores no necesariamente expertos en tecnología.
\end{itemize}

Este TFG se enfoca en el diseño e implementación de un prototipo que dé respuesta a estos desafíos, proporcionando una solución técnica robusta, funcional y bien documentada.

\section{Objetivos}
\label{sec:intro_objetivos}

Para abordar el problema definido, se establecen los siguientes objetivos, diferenciando entre el objetivo general y los específicos:

\subsection{Objetivo General}
\label{subsec:obj_general}

Diseñar e implementar un prototipo de sistema software para la monitorización remota de indicadores de salud de personas mayores, utilizando datos obtenidos de pulseras de actividad Fitbit\textsuperscript{\textregistered} y presentando la información de forma útil y accesible para cuidadores o personal autorizado, con un enfoque en la seguridad, la privacidad y la escalabilidad.

\subsection{Objetivos Específicos}
\label{subsec:obj_especificos}

\begin{enumerate}
    \item Desarrollar una arquitectura software modular basada en microservicios Python (Flask) que integre de forma segura la API de Fitbit\textsuperscript{\textregistered} mediante OAuth 2.0.
    \item Implementar la adquisición automática y el almacenamiento eficiente de datos biométricos y de actividad en una base de datos PostgreSQL con extensión TimescaleDB.
    \item Desarrollar un panel web intuitivo que permita visualizar el histórico de datos mediante gráficos interactivos y gestionar usuarios monitorizados.
    \item Implementar un sistema de alertas basado en evidencia para la detección temprana de cambios significativos en patrones de actividad, sueño y frecuencia cardíaca.
    \item Asegurar el cumplimiento del RGPD mediante medidas técnicas y organizativas apropiadas para la protección de datos personales de salud.
    \item Validar el funcionamiento del prototipo mediante pruebas funcionales, de integración y del sistema.
\end{enumerate}

\section{Alcance y Limitaciones}
\label{sec:intro_alcance}

El sistema desarrollado es un prototipo funcional que incluye:

\begin{itemize}
    \item Integración completa con la API de Fitbit\textsuperscript{\textregistered} para datos de frecuencia cardíaca, sueño y actividad física.
    \item Backend modular en Python con almacenamiento en PostgreSQL/TimescaleDB.
    \item Interfaz web para visualización de datos y gestión de usuarios.
    \item Sistema de alertas basado en criterios predefinidos.
    \item Implementación de medidas de seguridad y privacidad según RGPD.
\end{itemize}

Limitaciones principales:

\begin{itemize}
    \item No es un dispositivo médico certificado; su propósito es informativo y de apoyo al cuidado.
    \item Funcionalidad limitada a los datos disponibles vía API de Fitbit\textsuperscript{\textregistered}.
    \item Sistema de alertas basado en criterios iniciales, sin mecanismos avanzados de IA.
    \item Prototipo validado en entorno de desarrollo, pendiente de pruebas extensivas de carga y usabilidad.
\end{itemize}

\section{Estructura del Documento}
\label{sec:intro_estructura}

La memoria se organiza en los siguientes capítulos:

\begin{itemize}
    \item \textbf{Capítulo 1: Introducción.} (Este capítulo) Presenta el contexto, la motivación, el problema a resolver, los objetivos, el alcance y la estructura del documento.
    \item \textbf{Capítulo 2: Estado del Arte y Marco Tecnológico.} Revisa soluciones existentes en el ámbito de la monitorización remota de salud con wearables y describe en detalle las tecnologías clave seleccionadas y empleadas en el proyecto (Fitbit API, OAuth 2.0, microservicios, bases de datos de series temporales, etc.).
    \item \textbf{Capítulo 3: Análisis y Metodología.} Detalla los requisitos funcionales (lo que el sistema debe hacer) y no funcionales (atributos de calidad como rendimiento, seguridad, usabilidad) identificados para el sistema, y describe brevemente la metodología de desarrollo seguida (ej. iterativa, basada en prototipos).
    \item \textbf{Capítulo 4: Diseño y Arquitectura del Sistema.} Expone las decisiones de diseño tomadas, presentando la arquitectura general del sistema, el diseño detallado de los microservicios del backend, el esquema de la base de datos, el flujo de datos y la integración con la API externa.
    \item \textbf{Capítulo 5: Implementación.} Describe los detalles concretos de la implementación de los componentes más relevantes del sistema, incluyendo el entorno de desarrollo, las librerías principales utilizadas, fragmentos de código ilustrativos y los desafíos técnicos encontrados y cómo fueron resueltos.
    \item \textbf{Capítulo 6: Pruebas y Validación.} Explica la estrategia de pruebas definida y llevada a cabo (pruebas unitarias, de integración, del sistema) para asegurar la calidad del software y validar que el prototipo cumple con los requisitos especificados.
    \item \textbf{Capítulo 7: Resultados y Discusión.} Presenta el prototipo funcional resultante, mostrando ejemplos de su operación (ej. capturas de pantalla del dashboard) y discute los resultados obtenidos en términos de cumplimiento de objetivos, rendimiento observado y las limitaciones inherentes al sistema desarrollado.
    \item \textbf{Capítulo 8: Conclusiones y Trabajo Futuro.} Resume las principales conclusiones extraídas del desarrollo del TFG, destacando las contribuciones del trabajo y proponiendo posibles líneas de mejora, expansión y trabajo futuro sobre el sistema desarrollado.
\end{itemize}