% -*- coding: utf-8 -*-
\chapter{Estado del Arte y Marco Tecnológico}
\label{chap:estado_arte}

Este capítulo revisa el estado actual de la monitorización remota de salud en personas mayores, el papel de los dispositivos vestibles como Fitbit\textsuperscript{\textregistered}, y describe las tecnologías clave empleadas en el sistema desarrollado, justificando su elección.

\section{Monitorización Remota de Salud en Personas Mayores}
\label{sec:ea_monitorizacion_remota}

La monitorización remota de pacientes (RPM, por sus siglas en inglés, \textbf{Remote Patient Monitoring}) ha ganado relevancia en los últimos años, impulsada por los avances tecnológicos y la necesidad de modelos de atención sanitaria más eficientes \cite{noah2022mobile}. En personas mayores, la RPM permite la detección temprana de deterioros, la reducción de hospitalizaciones y la mejora de la independencia y la tranquilidad de usuarios y cuidadores \cite{bashshur2018telemedicine}.

Existen diversas aproximaciones a la RPM, desde sistemas basados en sensores ambientales instalados en el hogar hasta el uso de dispositivos médicos específicos o, cada vez más, el aprovechamiento de dispositivos de consumo como smartphones y wearables \cite{majumder2017wearable}. Sin embargo, la implementación exitosa de sistemas RPM para la población mayor también enfrenta desafíos importantes. Entre ellos destacan la usabilidad y aceptación de la tecnología por parte de los usuarios mayores, la gestión de la gran cantidad de datos generados, la necesidad de garantizar la fiabilidad y precisión de los datos, la interoperabilidad entre diferentes dispositivos y sistemas, y la gestión de la privacidad y seguridad de datos de salud altamente sensibles \cite{lee2021challenges}.

\subsection{Bases científicas y enfoque práctico para la detección de alertas}
\label{subsec:bases_alertas}

Más allá de la simple visualización, el valor añadido de la monitorización remota reside en la detección temprana de cambios relevantes en la salud del usuario. En este sistema, la selección de parámetros y la lógica de alerta se fundamentan en la literatura científica, pero se implementan mediante reglas y umbrales explícitos, adaptados a la práctica real y a las limitaciones de los datos disponibles por Fitbit. Los principales parámetros monitorizados son:

\begin{itemize}
    \item \textbf{Actividad física:} Se monitorizan caídas porcentuales significativas respecto a la media semanal individual, siguiendo la evidencia de que reducciones del 30\% o 50\% en la actividad habitual pueden indicar deterioro funcional o riesgo de eventos adversos \cite{rebelo_physical_inactivity_consequences_2020, who_guidelines_2020}. El sistema genera alertas cuando la actividad diaria cae por debajo de estos umbrales, calculados dinámicamente para cada usuario.
    \item \textbf{Sedentarismo:} Se detectan aumentos notables en el tiempo sedentario, comparando el valor diario con la línea base personal. Umbrales del 30\% y 50\% de incremento se emplean para alertas de prioridad media y alta, respectivamente, en línea con estudios recientes \cite{bellettiere_pa_sedentary_aging_women_2017}.
    \item \textbf{Sueño:} Se evalúan cambios en la duración del sueño respecto al patrón habitual del usuario. Variaciones superiores al 30\% (tanto por exceso como por defecto) generan alertas, dado su vínculo con deterioro cognitivo y fragilidad \cite{nsf_older_adult_sleep_2022, westman_sleep_cognition_nordic_2024, pinto_sleep_frailty_review_2025}.
    \item \textbf{Frecuencia cardíaca:} Se monitorizan anomalías en la frecuencia cardíaca en reposo (RHR) y su variabilidad (HRV), comparando los valores diarios con la media y desviación estándar individual. Se generan alertas si se superan umbrales relativos (por ejemplo, más de 2 desviaciones estándar respecto a la media semanal), siguiendo recomendaciones de la literatura \cite{kang_hrv_thresholds_mortality_2021, shaffer_overview_hrv_2017}.
\end{itemize}

Todos los umbrales se documentan y justifican en los anexos técnicos, y se calculan de forma personalizada para cada usuario.

\subsection{Sistemas de alerta en salud digital: enfoque implementado y efectividad}
\label{subsec:sistemas_alerta_comparativa}

La generación de alertas en este sistema se basa en un enfoque de reglas y umbrales explícitos, implementados en el backend (véase \texttt{alert\_rules.py}), por ser la opción más transparente y trazable para entornos clínicos y de cuidado. Este método, ampliamente utilizado en salud digital \cite{alam_alert_systems_review_2019}, permite adaptar fácilmente los criterios a nuevas evidencias o necesidades del usuario. Aunque existen enfoques más complejos (como modelos predictivos o integración de múltiples fuentes), en este TFG se prioriza la robustez, la interpretabilidad y la facilidad de validación.

Las alertas generadas se notifican a los cuidadores a través del panel web, donde pueden consultarse, filtrar por prioridad y marcar como revisadas.

\subsubsection{Efectividad y limitaciones del enfoque}
La efectividad de los sistemas de alerta depende tanto de la calidad de los datos como de la calibración de los umbrales. El sistema implementado busca minimizar la fatiga de alertas mediante la personalización de los umbrales y la priorización de alertas relevantes, pero reconoce limitaciones inherentes: posibles falsas alarmas si los patrones individuales varían mucho, y la imposibilidad de detectar eventos no reflejados en los datos de Fitbit. La validación empírica y la revisión periódica de los umbrales son esenciales para mantener la utilidad clínica del sistema.

\subsubsection{Justificación y validación de umbrales}
La definición de umbrales se basa en la literatura científica y en la experiencia clínica, pero se adapta a la variabilidad interindividual mediante el uso de porcentajes y comparación con la línea base personal. La justificación detallada de cada umbral y ventana temporal utilizada se documenta en los archivos técnicos del proyecto y se aborda en los capítulos de metodología e implementación.

\section{Dispositivos Wearables: El Caso de Fitbit}
\label{sec:ea_fitbit}

El mercado de dispositivos wearables ha experimentado un crecimiento exponencial, ofreciendo una amplia gama de productos capaces de monitorizar diversos parámetros fisiológicos y de actividad \cite{fortune_wearable_market}. Fitbit\textsuperscript{\textregistered} (ahora parte de Google) se ha consolidado como una de las marcas líderes en el segmento de pulseras y relojes de actividad física y bienestar. Sus dispositivos suelen incluir sensores como acelerómetros (para contar pasos y detectar movimiento/sueño) y fotopletismógrafos (PPG) para medir la frecuencia cardíaca \cite{fitbit_how_hr_works}.

Los datos típicamente accesibles a través de la API de Fitbit\textsuperscript{\textregistered} incluyen resúmenes diarios y, en algunos casos, datos intradía (con granularidad variable) de pasos, distancia, calorías quemadas, minutos de actividad, fases y duración del sueño, y frecuencia cardíaca \cite{fitbit_api_reference}. Si bien estos dispositivos no son instrumentos médicos certificados, diversos estudios han evaluado su precisión. Por ejemplo, la medición de la frecuencia cardíaca en reposo suele considerarse razonablemente precisa, aunque puede disminuir durante actividad física intensa. La detección de fases del sueño y el conteo de pasos también muestran una correlación aceptable con métodos de referencia en muchos estudios, aunque existen limitaciones y variabilidad entre dispositivos y condiciones de uso \cite{haghayegh2019accuracy, nelson2016validity}. Es crucial tener en cuenta estas consideraciones al interpretar los datos y diseñar el sistema. La adquisición de Fitbit por Google puede afectar la disponibilidad futura de la API \cite{google_fitbit_acquisition_info}.
\section{Tecnologías Habilitadoras}
\label{sec:ea_tecnologias}

El desarrollo del sistema de monitorización propuesto se apoya en un conjunto de tecnologías clave que se describen a continuación.

\subsection{API de Fitbit y OAuth 2.0}
\label{subsec:ea_fitbit_api_oauth}

El acceso a los datos de los usuarios de Fitbit\textsuperscript{\textregistered} se realiza exclusivamente a través de su API web oficial. Se trata de una API RESTful que utiliza el formato JSON para el intercambio de datos \cite{fitbit_api_reference}. Proporciona diversos \textit{endpoints} para obtener información del perfil del usuario, resúmenes de actividad diaria, datos de series temporales (como frecuencia cardíaca o pasos a lo largo del día con cierta granularidad), información sobre el sueño, etc. Para poder acceder a los datos de un usuario, es imprescindible obtener su consentimiento explícito a través del protocolo de autorización estándar \textbf{OAuth 2.0} \cite{oauth_spec_rfc6749}.

En este proyecto, se implementa el flujo \textit{Authorization Code Grant} de OAuth 2.0, considerado el más seguro para aplicaciones web con backend. Dado que la aplicación web desarrollada está pensada para ser operada por personal autorizado (ej. un recepcionista o cuidador en una residencia) y no directamente por el usuario final (la persona mayor), el flujo de vinculación se adapta ligeramente:
\begin{enumerate}
    \item El personal autorizado inicia sesión en la aplicación web del sistema de monitorización.
    \item Dentro de la aplicación, selecciona al residente o usuario final cuya cuenta de Fitbit\textsuperscript{\textregistered} desea vincular (identificado por su nombre y correo electrónico asociado a Fitbit\textsuperscript{\textregistered}).
    \item La aplicación web redirige el navegador del personal autorizado a la página de inicio de sesión y autorización de Fitbit\textsuperscript{\textregistered}, indicando los permisos (scopes) específicos que la aplicación necesita (ej. leer datos de frecuencia cardíaca, leer datos de actividad).
    \item El personal autorizado (o, idealmente, el propio residente si está presente y puede hacerlo) introduce las credenciales de la cuenta de Fitbit\textsuperscript{\textregistered} del residente y autoriza explícitamente a la aplicación a acceder a los datos solicitados en nombre de ese residente.
    \item Fitbit\textsuperscript{\textregistered} redirige el navegador de vuelta a la aplicación web del sistema, incluyendo un código de autorización temporal en la URL de redirección.
    \item El backend de la aplicación web recibe este código de autorización. De forma segura y sin exponerlo al navegador, intercambia este código (junto con las credenciales de la aplicación cliente registrada en Fitbit\textsuperscript{\textregistered}) directamente con el servidor de autorización de Fitbit\textsuperscript{\textregistered} para obtener un token de acceso (Access Token) y un token de refresco (Refresh Token) asociados a la cuenta del residente.
    \item El token de acceso se almacena de forma segura asociado al residente y se utiliza para realizar las llamadas posteriores a la API para obtener sus datos. Estos tokens tienen una vida útil limitada (ej. 8 horas).
    \item Cuando el token de acceso expira, el backend utiliza el token de refresco para obtener nuevos tokens automáticamente.
\end{enumerate}
La correcta y segura gestión de estos tokens (almacenamiento cifrado o seguro, uso exclusivo en el backend, uso de HTTPS en todas las comunicaciones) es fundamental para la seguridad y privacidad del sistema \cite{oauth_security_bcp_rfc8252}.

\subsection{Arquitecturas de Microservicios}
\label{subsec:ea_microservicios}

Frente a las arquitecturas monolíticas tradicionales, donde toda la funcionalidad de la aplicación reside en un único proceso desplegable, la arquitectura de microservicios estructura la aplicación como una colección de servicios pequeños, autónomos y débilmente acoplados \cite{fowler_microservices}. Cada servicio se centra en una capacidad de negocio específica, se comunica con otros servicios a través de APIs bien definidas (normalmente sobre HTTP/REST o colas de mensajes) y puede ser desarrollado, desplegado y escalado de forma independiente \cite{newman_building_microservices}.

Las ventajas clave de este enfoque, relevantes para nuestro sistema, incluyen:
\begin{itemize}
    \item \textbf{Escalabilidad Independiente:} Cada servicio puede escalarse horizontalmente según sus necesidades específicas (ej. escalar más instancias del servicio de adquisición de datos si hay muchos usuarios).
    \item \textbf{Resiliencia:} Un fallo en un servicio no tiene por qué detener todo el sistema; otros servicios pueden seguir funcionando (con mecanismos de tolerancia a fallos como \textit{circuit breakers}).
    \item \textbf{Flexibilidad Tecnológica:} Cada servicio puede desarrollarse con la tecnología más adecuada para su tarea específica (diferentes lenguajes, bases de datos).
    \item \textbf{Despliegue Independiente:} Los cambios en un servicio pueden desplegarse sin necesidad de redesplegar todo el sistema, agilizando las actualizaciones.
\end{itemize}
Sin embargo, los microservicios también introducen complejidad en áreas como la gestión distribuida de datos, la monitorización de múltiples servicios, el despliegue orquestado (donde herramientas como Docker y Kubernetes son muy útiles) y la necesidad de una cultura DevOps madura \cite{newman_building_microservices}. Para este TFG, se adopta un enfoque pragmático, diseñando un número limitado de microservicios bien definidos para aprovechar las ventajas de modularidad y escalabilidad potencial.

\subsection{Bases de Datos de Series Temporales}
\label{subsec:ea_db_timeseries}

Los datos generados por dispositivos wearables como Fitbit\textsuperscript{\textregistered} son inherentemente datos de series temporales: secuencias de mediciones indexadas por tiempo (timestamp). Si bien es posible almacenar estos datos en bases de datos relacionales tradicionales (como PostgreSQL o MySQL), las bases de datos especializadas en series temporales (TSDB - Time Series Databases) están optimizadas para este tipo de carga de trabajo \cite{dbengines_timeseries_ranking}.

Las TSDB suelen ofrecer ventajas significativas para datos de series temporales, como:
\begin{itemize}
    \item \textbf{Alto Rendimiento en Ingesta:} Optimizadas para escribir grandes volúmenes de datos nuevos secuencialmente en el tiempo.
    \item \textbf{Consultas Eficientes Basadas en Tiempo:} Indexación y funciones específicas para agregar, muestrear o filtrar datos por rangos de tiempo de forma muy rápida.
    \item \textbf{Compresión de Datos:} Técnicas específicas para comprimir datos temporales, que suelen tener cierta redundancia o patrones, ahorrando espacio de almacenamiento.
    \item \textbf{Políticas de Retención de Datos:} Facilidades para descartar automáticamente datos antiguos que ya no son necesarios (ej. mantener datos con granularidad de minutos por 1 mes, pero solo resúmenes diarios después de eso).
\end{itemize}
Ejemplos populares de TSDB incluyen InfluxDB y TimescaleDB (una extensión para PostgreSQL) \cite{influxdb_docs, timescaledb_docs}. Para este proyecto, se optó por \textbf{TimescaleDB} debido a su integración nativa con PostgreSQL, lo que permite combinar las ventajas de una TSDB con las capacidades de una base de datos relacional robusta, su uso de SQL estándar para las consultas y su madurez como proyecto \cite{timescaledb_docs}.

\subsection{Herramientas de Backend y Procesamiento}
\label{subsec:ea_backend_tools}

El backend del sistema, responsable de orquestar la autenticación, la adquisición de datos, el procesamiento y la exposición de APIs internas o para el frontend, se ha desarrollado utilizando \textbf{Python}. Python es una elección popular para el desarrollo web y el procesamiento de datos debido a su sintaxis clara, su amplio ecosistema de librerías y su gran comunidad \cite{python_website}.

Como framework web ligero para construir las APIs de los microservicios, se ha empleado \textbf{Flask} \cite{flask_docs}. Flask es un microframework que proporciona las herramientas básicas para el enrutamiento de peticiones HTTP y la gestión de respuestas, permitiendo una gran flexibilidad para elegir y añadir otras librerías según sea necesario (ej. para la interacción con la base de datos, serialización de datos, etc.). Esto se alinea con la filosofía de microservicios, manteniendo cada componente lo más ligero posible.

Para la adquisición periódica de datos desde la API de Fitbit\textsuperscript{\textregistered} (una tarea que debe ejecutarse de forma programada en segundo plano para cada usuario vinculado), se utiliza la librería \textbf{APScheduler} \cite{apscheduler_docs}. APScheduler permite definir trabajos (jobs) que se ejecutan en intervalos fijos, en fechas específicas o según expresiones cron, siendo adecuada para tareas de planificación dentro de una aplicación Python, como la consulta periódica a la API de Fitbit para cada usuario registrado.

\subsection{Tecnologías de Frontend/Visualización}
\label{subsec:ea_frontend_viz}

Para presentar la información monitorizada de forma clara y útil a los cuidadores, se ha desarrollado un panel web sencillo e intuitivo. Dada la base tecnológica en Python/Flask y la naturaleza de los datos (series temporales, gráficos estadísticos), la interfaz se construye con plantillas HTML, CSS y JavaScript integradas en Flask. La visualización se realiza mediante librerías JavaScript como Chart.js, permitiendo mostrar líneas de tiempo, resúmenes e indicadores clave de manera interactiva y comprensible. Esta solución prioriza la simplicidad y la mantenibilidad.

\section{Consideraciones Éticas y Legales (RGPD)}
\label{sec:ea_rgpd}

El sistema implementa los principios fundamentales del RGPD: consentimiento explícito mediante OAuth 2.0, uso limitado y minimizado de los datos, exactitud y conservación adecuada, y medidas técnicas y organizativas para garantizar la seguridad y confidencialidad (como HTTPS, almacenamiento seguro de credenciales y control de accesos). Se facilita el ejercicio de los derechos de los usuarios (acceso, rectificación, supresión, etc.) mediante mecanismos accesibles en la propia aplicación. La documentación de políticas y registros de consentimiento permite demostrar el cumplimiento normativo.