% -*- coding: utf-8 -*-
\chapter{Pruebas y Validación}
\label{chap:pruebas_validacion}

Este capítulo presenta la estrategia de validación del sistema y los resultados más relevantes de las pruebas realizadas. Los detalles técnicos completos, incluyendo código, configuraciones y resultados detallados, se encuentran en el Anexo \ref{anexo:pruebas}.

\section{Estrategia de Validación}
\label{sec:estrategia_validacion}

La validación del sistema se ha estructurado en tres niveles principales:

\begin{enumerate}
    \item \textbf{Pruebas Unitarias y de Integración:} Validación automatizada de componentes críticos, especialmente el sistema de alertas y la integración con la base de datos TimescaleDB.
    \item \textbf{Pruebas de Rendimiento:} Medición y validación de tiempos de respuesta y eficiencia en operaciones clave.
    \item \textbf{Pruebas con Datos Reales:} Validación del sistema con datos de tres dispositivos Fitbit reales durante un período de 30 días.
\end{enumerate}

\section{Resultados Principales}
\label{sec:resultados_principales}

\subsection{Pruebas de Carga}
Las pruebas de carga simularon 50 usuarios concurrentes durante 30 días, obteniendo:
\begin{itemize}
    \item Tiempo promedio de procesamiento por usuario: 0.85 segundos
    \item Tiempo máximo de procesamiento: 1.75 segundos
    \item Sin errores de concurrencia o pérdida de datos
\end{itemize}

\subsection{Validación de Umbrales}
Utilizando datos reales de tres dispositivos durante 30 días:
\begin{itemize}
    \item Tasa de falsos positivos: < 5\%
    \item Detección correcta de patrones anómalos: > 95\%
    \item Distribución equilibrada de alertas entre días laborables y fines de semana
\end{itemize}

\subsection{Rendimiento del Sistema}
Métricas clave de rendimiento:
\begin{itemize}
    \item Dashboard: Carga inicial < 2 segundos
    \item API interna: Endpoints críticos < 500ms
    \item Base de datos: Consultas optimizadas (60-120ms)
    \item Procesamiento de alertas: Tiempo real (< 200ms/usuario)
\end{itemize}

\section{Áreas de Mejora Identificadas}
\label{sec:areas_mejora}

Las pruebas han identificado las siguientes áreas para futuro desarrollo:

\begin{itemize}
    \item \textbf{Escalabilidad:} Aunque el sistema maneja bien 50 usuarios concurrentes, se recomienda implementar caché y optimizaciones adicionales para mayor escala.
    \item \textbf{Validación de Umbrales:} Los umbrales actuales funcionan bien pero podrían beneficiarse de ajuste fino basado en más datos reales.
    \item \textbf{Monitorización:} Implementar un sistema de monitorización en tiempo real para detectar y responder a problemas de rendimiento.
\end{itemize}

Los detalles completos de las pruebas, incluyendo configuraciones, casos de prueba específicos y resultados detallados, se encuentran documentados en el Anexo \ref{anexo:pruebas}.


