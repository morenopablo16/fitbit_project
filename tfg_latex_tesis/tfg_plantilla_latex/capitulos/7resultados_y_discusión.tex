% -*- coding: utf-8 -*-
\chapter{Resultados y Discusión}
\label{chap:resultados_discusion}

Este capítulo presenta y analiza los resultados obtenidos del sistema de monitorización desarrollado. Se estructura en cuatro secciones principales: (1) la validación funcional del prototipo, (2) los resultados de las pruebas de rendimiento y métricas del sistema, (3) la evaluación del sistema de alertas, y (4) una discusión crítica que conecta los resultados con los objetivos iniciales y analiza las limitaciones del trabajo.

\section{Validación Funcional del Prototipo}
\label{sec:validacion_funcional}

El prototipo desarrollado implementa satisfactoriamente los requisitos funcionales definidos en el Capítulo \ref{chap:requisitos_metodologia}. A continuación, se presenta la evidencia de las funcionalidades clave implementadas:

\begin{itemize}
    \item \textbf{Vinculación y Gestión de Dispositivos:} El sistema implementa correctamente el flujo OAuth 2.0 con PKCE para la vinculación de dispositivos Fitbit\textsuperscript{\textregistered}, incluyendo:
        \begin{itemize}
            \item Gestión segura de tokens (cifrado en base de datos)
            \item Refresco automático de tokens expirados
            \item Reasignación de dispositivos entre usuarios
        \end{itemize}
    \item \textbf{Adquisición y Almacenamiento de Datos:} Los scripts de adquisición (\texttt{fitbit.py}, \texttt{fitbit\_intraday.py}) ejecutados por cron obtienen y almacenan correctamente:
        \begin{itemize}
            \item Datos diarios: pasos, frecuencia cardíaca, sueño, actividad
            \item Datos intradía: métricas con granularidad por minuto/hora
            \item Persistencia en TimescaleDB con timestamps precisos
        \end{itemize}
    \item \textbf{Visualización y Dashboard:} La interfaz web implementa:
        \begin{itemize}
            \item Gráficos interactivos de series temporales
            \item Filtros por fecha y tipo de dato
            \item Exportación de datos en formato CSV
            \item Interfaz multilingüe (español/inglés)
        \end{itemize}
\end{itemize}

\section{Rendimiento y Métricas del Sistema}
\label{sec:rendimiento_metricas}

Las pruebas de rendimiento se realizaron en el siguiente entorno:
\begin{itemize}
    \item \textbf{Hardware:} Máquina virtual con 4GB RAM, 2 vCPUs
    \item \textbf{Software:} PostgreSQL 13 con TimescaleDB, Python 3.8, Flask 2.0
    \item \textbf{Datos:} 
        \begin{itemize}
            \item Datos reales de 3 usuarios con dispositivos Fitbit (el autor y dos colaboradores del TFG), recopilados durante 30 días
            \item Datos simulados adicionales para pruebas específicas de rendimiento y validación
        \end{itemize}
\end{itemize}

Los resultados cuantitativos obtenidos son:

\begin{itemize}
    \item \textbf{Tiempos de Respuesta:}
        \begin{itemize}
            \item Dashboard (carga inicial): 1.8 segundos promedio
            \item API interna: 280-450ms promedio en endpoints críticos
            \item Generación de alertas: 200ms por usuario
        \end{itemize}
    \item \textbf{Eficiencia de Base de Datos:}
        \begin{itemize}
            \item Consultas a TimescaleDB: 60-120ms para operaciones típicas
            \item Uso de memoria: 420MB en operación normal
            \item Compresión TimescaleDB: 58\% de reducción en series temporales
        \end{itemize}
\end{itemize}

\section{Evaluación del Sistema de Alertas}
\label{sec:evaluacion_alertas}

La evaluación del sistema de alertas se realizó mediante dos aproximaciones complementarias:

\begin{enumerate}
    \item \textbf{Validación con Datos Reales:}
        \begin{itemize}
            \item Monitorización de 3 usuarios reales durante 30 días
            \item Análisis de patrones de actividad, sueño y frecuencia cardíaca en condiciones reales de uso
            \item Verificación de la detección de eventos significativos (ej. días de baja actividad, alteraciones del sueño)
        \end{itemize}
    \item \textbf{Pruebas Automatizadas:}
        \begin{itemize}
            \item Test suite documentado en \texttt{test\_alerts\_full.py}
            \item Escenarios controlados con datos simulados
            \item Validación sistemática de la lógica de detección
        \end{itemize}
\end{enumerate}

El procedimiento de pruebas automatizadas incluyó:

\begin{itemize}
    \item \textbf{Línea Base:} 6 días de datos normales (10.000 pasos/día, 800 min sedentarios, etc.)
    \item \textbf{Anomalías Controladas:} Inserción de valores anómalos predefinidos
    \item \textbf{Validación:} Verificación de detección y priorización correcta
\end{itemize}

Los resultados combinados de ambas aproximaciones muestran:

\begin{itemize}
    \item \textbf{Detección:} El sistema identifica correctamente cambios significativos tanto en datos reales como simulados
    \item \textbf{Priorización:} Las alertas se clasifican adecuadamente según su severidad:
        \begin{itemize}
            \item Alta: Desviaciones >50\% o períodos críticos
            \item Media: Desviaciones 30-50\%
            \item Baja: Desviaciones 20-30\%
        \end{itemize}
\end{itemize}

\section{Discusión de Resultados}
\label{sec:discusion}

\subsection{Cumplimiento de Objetivos}
\label{subsec:cumplimiento_objetivos}

Revisando los objetivos específicos definidos en el Capítulo \ref{chap:introduccion}:

\begin{enumerate}
    \item \textbf{Integración con API Fitbit:} Implementada completamente, incluyendo:
        \begin{itemize}
            \item Flujo OAuth 2.0 con PKCE
            \item Gestión segura de tokens
            \item Adquisición automática de datos
        \end{itemize}
    \item \textbf{Arquitectura Modular:} Lograda mediante:
        \begin{itemize}
            \item Separación clara de componentes (auth, db, fitbit)
            \item Scripts independientes para adquisición
            \item Interfaz web desacoplada
        \end{itemize}
    \item \textbf{Sistema de Alertas:} Implementado con:
        \begin{itemize}
            \item Criterios basados en evidencia científica
            \item Personalización por usuario
            \item Priorización automática
        \end{itemize}
\end{enumerate}

\subsection{Limitaciones del Trabajo}
\label{subsec:limitaciones}

Es importante reconocer las siguientes limitaciones del prototipo actual:

\begin{itemize}
    \item \textbf{Validación de Alertas:}
        \begin{itemize}
            \item Muestra limitada de usuarios reales (3)
            \item Sin validación clínica formal de los umbrales
            \item Período de observación relativamente corto (30 días)
        \end{itemize}
    \item \textbf{Rendimiento:}
        \begin{itemize}
            \item Pruebas en entorno controlado de desarrollo
            \item No se ha evaluado el comportamiento bajo alta carga
            \item Faltan pruebas de estrés y escalabilidad
        \end{itemize}
    \item \textbf{Funcionalidad:}
        \begin{itemize}
            \item Sin integración con sistemas clínicos externos
            \item Limitado a datos disponibles vía API Fitbit
            \item Sin mecanismos avanzados de predicción
        \end{itemize}
\end{itemize}

\subsection{Implicaciones Prácticas}
\label{subsec:implicaciones}

Los resultados sugieren que el sistema tiene potencial para:

\begin{itemize}
    \item \textbf{Monitorización Automatizada:} El sistema automatiza la recolección y análisis de datos de actividad, sueño y frecuencia cardíaca, reduciendo la necesidad de monitorización manual.
    \item \textbf{Detección Temprana:} La implementación de alertas, aunque preliminar, sienta las bases para un sistema de detección temprana de cambios significativos en patrones de actividad y biomarcadores.
    \item \textbf{Seguimiento Remoto:} La arquitectura web y el uso de dispositivos comerciales facilita el despliegue con infraestructura mínima.
\end{itemize}

\subsection{Lecciones Aprendidas y Desafíos}
\label{subsec:lecciones}

Durante el desarrollo se identificaron varios desafíos importantes:

\begin{itemize}
    \item \textbf{Calidad de Datos:} La variabilidad en el uso de los dispositivos y la sincronización afecta la consistencia de los datos.
    \item \textbf{Personalización:} El equilibrio entre alertas genéricas y umbrales personalizados requiere más investigación.
    \item \textbf{Escalabilidad:} La arquitectura actual podría requerir optimizaciones para manejar grandes volúmenes de usuarios.
\end{itemize}

Estas experiencias serán valiosas para futuras iteraciones del sistema o proyectos similares en el campo de la monitorización remota de salud.