% -*- coding: utf-8 -*-
\chapter{Conclusiones y Trabajo Futuro}
\label{chap:conclusiones}

Este capítulo final presenta las conclusiones extraídas del desarrollo del sistema de monitorización, evalúa el cumplimiento de los objetivos iniciales y propone líneas de trabajo futuro.

\section{Conclusiones Generales}
\label{sec:conclusiones_generales}

El desarrollo de este Trabajo Fin de Grado ha permitido demostrar la viabilidad de crear un sistema de monitorización remota basado en dispositivos Fitbit\textsuperscript{\textregistered}, alcanzando los siguientes logros:

\begin{itemize}
    \item Implementación exitosa de una plataforma completa de monitorización, desde la adquisición de datos hasta la visualización y generación de alertas.
    \item Desarrollo de una arquitectura modular y escalable basada en microservicios.
    \item Integración segura con la API de Fitbit\textsuperscript{\textregistered} mediante OAuth 2.0.
    \item Creación de un sistema de alertas personalizable y efectivo.
    \item Implementación de una interfaz web intuitiva y multilingüe.
\end{itemize}

\section{Cumplimiento de Objetivos}
\label{sec:cumplimiento_objetivos}

Revisando los objetivos planteados en el Capítulo \ref{chap:introduccion}:

\begin{itemize}
    \item \textbf{Objetivos Técnicos:}
        \begin{itemize}
            \item Integración completa con la API de Fitbit\textsuperscript{\textregistered}
            \item Implementación de una arquitectura robusta y modular
            \item Gestión eficiente de datos temporales con TimescaleDB
        \end{itemize}
    \item \textbf{Objetivos Funcionales:}
        \begin{itemize}
            \item Sistema de alertas basado en evidencia
            \item Visualización efectiva de datos
            \item Gestión segura de usuarios y datos
        \end{itemize}
\end{itemize}

\section{Contribuciones Principales}
\label{sec:contribuciones}

Las principales contribuciones de este trabajo incluyen:

\begin{itemize}
    \item Una arquitectura de referencia para sistemas de monitorización remota, combinando seguridad, escalabilidad y usabilidad
    \item Un modelo de implementación para la gestión segura de datos de salud cumpliendo con RGPD
    \item Un sistema de alertas adaptativo basado en evidencia científica
    \item Una base de código documentada y reutilizable disponible en código abierto
\end{itemize}

\section{Limitaciones y Desafíos}
\label{sec:limitaciones}

Las principales limitaciones identificadas son:

\begin{itemize}
    \item \textbf{Técnicas:}
        \begin{itemize}
            \item Dependencia de la disponibilidad y límites de la API de Fitbit\textsuperscript{\textregistered}
            \item Granularidad limitada en ciertos tipos de datos
            \item Necesidad de optimización para grandes volúmenes de usuarios
        \end{itemize}
    \item \textbf{Funcionales:}
        \begin{itemize}
            \item Falta de validación clínica exhaustiva
            \item Ausencia de integración con sistemas de salud existentes
            \item Limitaciones en la personalización de alertas por perfil médico
        \end{itemize}
\end{itemize}

\section{Líneas de Trabajo Futuro}
\label{sec:trabajo_futuro}

Se identifican las siguientes líneas prioritarias de desarrollo:

\begin{itemize}
    \item \textbf{Mejoras Técnicas:}
        \begin{itemize}
            \item Implementación de análisis predictivo con machine learning
            \item Soporte para múltiples tipos de dispositivos wearables
            \item Desarrollo de una API REST pública para integración
        \end{itemize}
    \item \textbf{Validación y Extensión:}
        \begin{itemize}
            \item Estudios piloto en entornos reales
            \item Validación de umbrales y criterios de alerta
            \item Evaluación del impacto en la atención sanitaria
        \end{itemize}
    \item \textbf{Extensiones Funcionales:}
        \begin{itemize}
            \item Desarrollo de una aplicación móvil complementaria
            \item Integración con sistemas de historia clínica electrónica
            \item Implementación de análisis de tendencias a largo plazo
        \end{itemize}
\end{itemize}

\section{Reflexión Final}
\label{sec:reflexion_final}

Este TFG ha demostrado el potencial de los dispositivos comerciales como Fitbit\textsuperscript{\textregistered} para la monitorización remota de personas mayores. La combinación de tecnologías modernas, arquitecturas escalables y criterios clínicos puede resultar en herramientas valiosas para el cuidado de la salud.

El trabajo realizado sienta las bases para futuros desarrollos en el campo de la monitorización remota, proporcionando una arquitectura de referencia y lecciones aprendidas que pueden beneficiar a proyectos similares.
