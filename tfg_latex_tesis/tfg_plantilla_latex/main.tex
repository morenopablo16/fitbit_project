
\documentclass[12pt, a4paper, oneside]{book}
\usepackage{listings}
\usepackage{color} 
\lstset{
    inputencoding=utf8,         % <-- MUY IMPORTANTE
    extendedchars=true,         % <-- Activar caracteres extendidos (acentos, ñ, etc.)
    literate=%
        {á}{{\'a}}1 {é}{{\'e}}1 {í}{{\'i}}1 {ó}{{\'o}}1 {ú}{{\'u}}1
        {Á}{{\'A}}1 {É}{{\'E}}1 {Í}{{\'I}}1 {Ó}{{\'O}}1 {Ú}{{\'U}}1
        {ñ}{{\~n}}1 {Ñ}{{\~N}}1
        {¿}{{?}}1 {¡}{{!}}1,
    breaklines=true,
    breakatwhitespace=true,
    columns=flexible,
    basicstyle=\ttfamily\small,
    numbers=left,
    numberstyle=\tiny\color{gray},
    frame=single,
    captionpos=b,
    keepspaces=true,
    tabsize=4,
}
\usepackage{tikz}
\usetikzlibrary{shapes,arrows,positioning}
\usepackage[spanish]{babel}
\usepackage[utf8]{inputenc}
\usepackage[T1]{fontenc}
\usepackage{geometry}
\geometry{a4paper, margin=2.5cm}

\usepackage{titlesec}
\usepackage{titletoc}
\usepackage{hyperref}
\usepackage{graphicx}
\usepackage{tabularx}
\usepackage{xcolor}
\usepackage{amsmath}
\usepackage{natbib}
\usepackage{appendix}

\hypersetup{
    colorlinks=true,
    linkcolor=blue,
    filecolor=magenta,      
    urlcolor=cyan,
    pdftitle={TFG Pablo Moreno Muñoz},
}

\titleformat{\chapter}[display]
{\normalfont\bfseries\Huge}{\chaptertitlename\ \thechapter}{20pt}{\Huge}
\titlespacing*{\chapter}{0pt}{-30pt}{40pt}

\setcounter{secnumdepth}{3}
\setcounter{tocdepth}{3}

\begin{document}

\begin{titlepage}
    \centering
    \vspace*{2cm}
    {\LARGE Universidad de Zaragoza \par}
    \vspace{1cm}
    {\Huge\bfseries Diseño e implementación de un sistema basado en wearables Fitbit para el monitoreo de la salud de personas mayores \par}
    \vspace{2cm}
    {\Large Pablo Moreno Muñoz \par}
    \vspace{1cm}
    {\large Trabajo Fin de Grado \par}
    \vspace{1cm}
    {\large Grado en Ingeniería Informática \par}
    \vspace{1cm}
    {\large Directores: Raquel Trillo y Laura Po \par}
    \vspace{2cm}
    {\large \today \par}
\end{titlepage}

\chapter*{Resumen}
 Este trabajo se enmarca en la creciente necesidad de soluciones tecnológicas para la atención sanitaria de la población anciana. Presentamos el diseño y la implementación de un sistema de monitorización remota de salud, apoyado en dispositivos Fitbit, que permite el seguimiento continuo de indicadores como la frecuencia cardíaca, los niveles de actividad física y la calidad del sueño.

El sistema desarrollado emplea un backend modular basado en microservicios, diseñado con criterios de escalabilidad y resiliencia. Se integra de manera segura con la API de Fitbit utilizando el protocolo OAuth 2.0, y permite la adquisición y almacenamiento eficiente de datos en bases de datos especializadas en series temporales.  Además, se incorpora un pipeline que procesa los datos de forma automatizada tras su recepción, permitiendo la detección de eventos críticos, y paneles visuales accesibles e intuitivos que permiten a cuidadores o profesionales de la salud visualizar la evolución del estado de los usuarios de forma clara.

La tesis expone los requerimientos técnicos, las decisiones arquitectónicas adoptadas, los desafíos enfrentados y las pruebas realizadas para validar el sistema. Asimismo, se analizan las implicaciones éticas y legales relacionadas con el tratamiento de datos personales de salud, cumpliendo con el Reglamento General de Protección de Datos (RGPD). Finalmente, se discuten las oportunidades de mejora y expansión del sistema, incluyendo la integración con nuevas plataformas de dispositivos vestibles y el uso de técnicas de inteligencia artificial para predicción de eventos adversos.

\vspace{1cm} % Espacio opcional
\textbf{Palabras clave:} Monitorización remota, Fitbit, Salud ancianos, OAuth 2.0. 

\tableofcontents

% -*- coding: utf-8 -*- % Para asegurar codificación correcta
\chapter{Introducción}
\label{chap:introduccion}
A continuación se introduce el contexto y la motivación de este proyecto, para luego definir sus objetivos, alcance y la organización general de este documento.

\section{Contexto y Motivación}
\label{sec:intro_contexto}

El envejecimiento de la población es una realidad demográfica global, especialmente acentuada en España, donde, según el Instituto Nacional de Estadística, más del 20\% de la población tiene más de 65 años y se prevé que esta cifra siga aumentando en las próximas décadas \cite{ine_proyeccion_2022_2072}. Este fenómeno implica un incremento en la prevalencia de enfermedades crónicas y una mayor demanda de servicios sanitarios y de cuidados de larga duración, lo que genera una presión significativa sobre los sistemas de salud y las familias \cite{who_ageing_health}.

La monitorización remota de la salud, apoyada en dispositivos vestibles como Fitbit\textsuperscript{\textregistered}, ofrece un potencial significativo para el seguimiento continuo y no invasivo de indicadores fisiológicos y de actividad \cite{majumder2017wearable}. Estos dispositivos, que se han popularizado por su facilidad de uso y coste accesible, generan datos valiosos que, adecuadamente procesados, pueden contribuir a una atención más proactiva y personalizada.

La motivación principal de este Trabajo Fin de Grado (TFG) es desarrollar un sistema que aproveche los datos de dispositivos Fitbit\textsuperscript{\textregistered} para facilitar la monitorización remota de personas mayores, permitiendo detectar cambios significativos en su estado de salud de forma temprana y mejorando la tranquilidad de cuidadores y familiares.

\section{Definición del Problema}
\label{sec:intro_problema}

A pesar de la disponibilidad de datos generados por dispositivos como Fitbit\textsuperscript{\textregistered}, existen dos desafíos principales que este trabajo busca abordar:

\begin{itemize}
    \item La necesidad de un sistema integrado que recopile automáticamente datos de Fitbit\textsuperscript{\textregistered} y los presente de forma clara y contextualizada para cuidadores, permitiendo un seguimiento longitudinal efectivo de indicadores clave de salud.
    \item La importancia de implementar una solución técnica robusta que gestione de forma segura la autenticación, respete la privacidad de los datos de salud según el RGPD, y permita una futura expansión del sistema.
    \item La falta de sistemas integrados que recopilen automáticamente datos relevantes de Fitbit\textsuperscript{\textregistered} y los presenten de forma clara, contextualizada y comprensible para cuidadores no necesariamente expertos en tecnología.
\end{itemize}

Este TFG se enfoca en el diseño e implementación de un prototipo que dé respuesta a estos desafíos, proporcionando una solución técnica robusta, funcional y bien documentada.

\section{Objetivos}
\label{sec:intro_objetivos}

Para abordar el problema definido, se establecen los siguientes objetivos, diferenciando entre el objetivo general y los específicos:

\subsection{Objetivo General}
\label{subsec:obj_general}

Diseñar e implementar un prototipo de sistema software para la monitorización remota de indicadores de salud de personas mayores, utilizando datos obtenidos de pulseras de actividad Fitbit\textsuperscript{\textregistered} y presentando la información de forma útil y accesible para cuidadores o personal autorizado, con un enfoque en la seguridad, la privacidad y la escalabilidad.

\subsection{Objetivos Específicos}
\label{subsec:obj_especificos}

\begin{enumerate}
    \item Desarrollar una arquitectura software modular basada en microservicios Python (Flask) que integre de forma segura la API de Fitbit\textsuperscript{\textregistered} mediante OAuth 2.0.
    \item Implementar la adquisición automática y el almacenamiento eficiente de datos biométricos y de actividad en una base de datos PostgreSQL con extensión TimescaleDB.
    \item Desarrollar un panel web intuitivo que permita visualizar el histórico de datos mediante gráficos interactivos y gestionar usuarios monitorizados.
    \item Implementar un sistema de alertas basado en evidencia para la detección temprana de cambios significativos en patrones de actividad, sueño y frecuencia cardíaca.
    \item Asegurar el cumplimiento del RGPD mediante medidas técnicas y organizativas apropiadas para la protección de datos personales de salud.
    \item Validar el funcionamiento del prototipo mediante pruebas funcionales, de integración y del sistema.
\end{enumerate}

\section{Alcance y Limitaciones}
\label{sec:intro_alcance}

El sistema desarrollado es un prototipo funcional que incluye:

\begin{itemize}
    \item Integración completa con la API de Fitbit\textsuperscript{\textregistered} para datos de frecuencia cardíaca, sueño y actividad física.
    \item Backend modular en Python con almacenamiento en PostgreSQL/TimescaleDB.
    \item Interfaz web para visualización de datos y gestión de usuarios.
    \item Sistema de alertas basado en criterios predefinidos.
    \item Implementación de medidas de seguridad y privacidad según RGPD.
\end{itemize}

Limitaciones principales:

\begin{itemize}
    \item No es un dispositivo médico certificado; su propósito es informativo y de apoyo al cuidado.
    \item Funcionalidad limitada a los datos disponibles vía API de Fitbit\textsuperscript{\textregistered}.
    \item Sistema de alertas basado en criterios iniciales, sin mecanismos avanzados de IA.
    \item Prototipo validado en entorno de desarrollo, pendiente de pruebas extensivas de carga y usabilidad.
\end{itemize}

\section{Estructura del Documento}
\label{sec:intro_estructura}

La memoria se organiza en los siguientes capítulos:

\begin{itemize}
    \item \textbf{Capítulo 1: Introducción.} (Este capítulo) Presenta el contexto, la motivación, el problema a resolver, los objetivos, el alcance y la estructura del documento.
    \item \textbf{Capítulo 2: Estado del Arte y Marco Tecnológico.} Revisa soluciones existentes en el ámbito de la monitorización remota de salud con wearables y describe en detalle las tecnologías clave seleccionadas y empleadas en el proyecto (Fitbit API, OAuth 2.0, microservicios, bases de datos de series temporales, etc.).
    \item \textbf{Capítulo 3: Análisis y Metodología.} Detalla los requisitos funcionales (lo que el sistema debe hacer) y no funcionales (atributos de calidad como rendimiento, seguridad, usabilidad) identificados para el sistema, y describe brevemente la metodología de desarrollo seguida (ej. iterativa, basada en prototipos).
    \item \textbf{Capítulo 4: Diseño y Arquitectura del Sistema.} Expone las decisiones de diseño tomadas, presentando la arquitectura general del sistema, el diseño detallado de los microservicios del backend, el esquema de la base de datos, el flujo de datos y la integración con la API externa.
    \item \textbf{Capítulo 5: Implementación.} Describe los detalles concretos de la implementación de los componentes más relevantes del sistema, incluyendo el entorno de desarrollo, las librerías principales utilizadas, fragmentos de código ilustrativos y los desafíos técnicos encontrados y cómo fueron resueltos.
    \item \textbf{Capítulo 6: Pruebas y Validación.} Explica la estrategia de pruebas definida y llevada a cabo (pruebas unitarias, de integración, del sistema) para asegurar la calidad del software y validar que el prototipo cumple con los requisitos especificados.
    \item \textbf{Capítulo 7: Resultados y Discusión.} Presenta el prototipo funcional resultante, mostrando ejemplos de su operación (ej. capturas de pantalla del dashboard) y discute los resultados obtenidos en términos de cumplimiento de objetivos, rendimiento observado y las limitaciones inherentes al sistema desarrollado.
    \item \textbf{Capítulo 8: Conclusiones y Trabajo Futuro.} Resume las principales conclusiones extraídas del desarrollo del TFG, destacando las contribuciones del trabajo y proponiendo posibles líneas de mejora, expansión y trabajo futuro sobre el sistema desarrollado.
\end{itemize}
% -*- coding: utf-8 -*-
\chapter{Estado del Arte y Marco Tecnológico}
\label{chap:estado_arte}

Este capítulo revisa el estado actual de la monitorización remota de salud en personas mayores, el papel de los dispositivos vestibles como Fitbit\textsuperscript{\textregistered}, y describe las tecnologías clave empleadas en el sistema desarrollado, justificando su elección.

\section{Monitorización Remota de Salud en Personas Mayores}
\label{sec:ea_monitorizacion_remota}

La monitorización remota de pacientes (RPM, por sus siglas en inglés, \textbf{Remote Patient Monitoring}) ha ganado relevancia impulsada por la necesidad de modelos de atención sanitaria más eficientes, especialmente para la población mayor \cite{noah2022mobile}. En España, donde más del 20\% de la población supera los 65 años \cite{ine_proyeccion_2022_2072}, la RPM permite la detección temprana de deterioros y la mejora de la independencia de usuarios y cuidadores \cite{bashshur2018telemedicine}.

Las aproximaciones a la RPM varían desde sensores ambientales hasta dispositivos médicos específicos o wearables de consumo \cite{majumder2017wearable}. Los principales desafíos incluyen la usabilidad para usuarios mayores, la gestión de datos masivos, la fiabilidad de las mediciones y la privacidad de datos sensibles \cite{lee2021challenges}.

\subsection{Bases científicas y enfoque práctico para la detección de alertas}
\label{subsec:bases_alertas}

El sistema implementa reglas y umbrales basados en evidencia científica, adaptados a las limitaciones de los datos disponibles por Fitbit. Los principales parámetros monitorizados incluyen actividad física (caídas del 30-50\% respecto a la media semanal) \cite{rebelo_physical_inactivity_consequences_2020, who_guidelines_2020}, sedentarismo \cite{bellettiere_pa_sedentary_aging_women_2017}, patrones de sueño \cite{nsf_older_adult_sleep_2022} y frecuencia cardíaca \cite{kang_hrv_thresholds_mortality_2021}. Los umbrales detallados y su justificación se documentan en el Anexo~\ref{anexo:umbrales_alertas}.

\subsection{Sistemas de alerta en salud digital}
\label{subsec:sistemas_alerta_comparativa}

La generación de alertas en este sistema se basa en un enfoque de reglas y umbrales explícitos, implementados en el backend (véase \texttt{alert\_rules.py}), por ser la opción más transparente y trazable para entornos clínicos y de cuidado. Este método, ampliamente utilizado en salud digital \cite{alam_alert_systems_review_2019}, permite adaptar fácilmente los criterios a nuevas evidencias o necesidades del usuario. Aunque existen enfoques más complejos (como modelos predictivos o integración de múltiples fuentes), en este TFG se prioriza la robustez, la interpretabilidad y la facilidad de validación.

Las alertas generadas se notifican a los cuidadores a través del panel web, donde pueden consultarse, filtrar por prioridad y marcar como revisadas.

\subsubsection{Efectividad y limitaciones del enfoque}
La efectividad de los sistemas de alerta depende tanto de la calidad de los datos como de la calibración de los umbrales. El sistema implementado busca minimizar la fatiga de alertas mediante la personalización de los umbrales y la priorización de alertas relevantes, pero reconoce limitaciones inherentes: posibles falsas alarmas si los patrones individuales varían mucho, y la imposibilidad de detectar eventos no reflejados en los datos de Fitbit. La validación empírica y la revisión periódica de los umbrales son esenciales para mantener la utilidad clínica del sistema.

\subsubsection{Justificación y validación de umbrales}
La definición de umbrales se basa en la literatura científica y en la experiencia clínica, pero se adapta a la variabilidad interindividual mediante el uso de porcentajes y comparación con la línea base personal. La justificación detallada de cada umbral y ventana temporal utilizada se documenta en los archivos técnicos del proyecto y se aborda en los capítulos de metodología e implementación.

\section{Dispositivos Wearables: El Caso de Fitbit}
\label{sec:ea_fitbit}

El mercado de dispositivos wearables ha experimentado un crecimiento exponencial \cite{fortune_wearable_market}. Fitbit\textsuperscript{\textregistered}, ahora parte de Google, es líder en pulseras y relojes de actividad física, con dispositivos que incluyen acelerómetros y fotopletismógrafos (PPG) para medir movimiento y frecuencia cardíaca \cite{fitbit_how_hr_works}.

Su API proporciona datos diarios e intradía de actividad física, sueño y frecuencia cardíaca \cite{fitbit_api_reference}. Aunque no son dispositivos médicos certificados, estudios validan su precisión aceptable en mediciones como frecuencia cardíaca en reposo, fases del sueño y conteo de pasos, con ciertas limitaciones según el dispositivo y condiciones de uso \cite{haghayegh2019accuracy, nelson2016validity}. Estas consideraciones son cruciales al interpretar los datos. La adquisición por Google podría afectar la disponibilidad futura de la API \cite{google_fitbit_acquisition_info}.
\section{Tecnologías Habilitadoras}
\label{sec:ea_tecnologias}

\subsection{API de Fitbit y OAuth 2.0}
\label{subsec:ea_fitbit_api_oauth}
El acceso a los datos de los usuarios de Fitbit\textsuperscript{\textregistered} se realiza exclusivamente a través de su API web oficial. Se trata de una API RESTful que utiliza el formato JSON para el intercambio de datos \cite{fitbit_api_reference}. Proporciona diversos \textit{endpoints} para obtener información del perfil del usuario, resúmenes de actividad diaria, datos de series temporales (como frecuencia cardíaca o pasos a lo largo del día con cierta granularidad), información sobre el sueño, etc. Para poder acceder a los datos de un usuario, es imprescindible obtener su consentimiento explícito a través del protocolo de autorización estándar \textbf{OAuth 2.0} \cite{oauth_spec_rfc6749}.

En este proyecto, se implementa el flujo \textit{Authorization Code Grant} de OAuth 2.0, considerado el más seguro para aplicaciones web con backend. Los detalles técnicos completos del proceso de autenticación y autorización, incluyendo diagramas de secuencia, ejemplos de respuestas de la API, y la gestión de tokens, se encuentran documentados en el Anexo~\ref{anexo:oatuh_fitbit}.
La correcta y segura gestión de estos tokens (almacenamiento cifrado o seguro, uso exclusivo en el backend, uso de HTTPS en todas las comunicaciones) es fundamental para la seguridad y privacidad del sistema \cite{oauth_security_bcp_rfc8252}.

\subsection{Arquitecturas de Microservicios}
\label{subsec:ea_microservicios}

Frente a las arquitecturas tradicionales, donde toda la funcionalidad de la aplicación reside en un único proceso desplegable, la arquitectura de microservicios estructura la aplicación como una colección de servicios pequeños, autónomos y débilmente acoplados \cite{fowler_microservices}. Cada servicio se centra en una capacidad de negocio específica, se comunica con otros servicios a través de APIs bien definidas (normalmente sobre HTTP/REST o colas de mensajes) y puede ser desarrollado, desplegado y escalado de forma independiente \cite{newman_building_microservices}.

Las ventajas clave de este enfoque, relevantes para nuestro sistema, incluyen:
\begin{itemize}
    \item \textbf{Escalabilidad Independiente:} Cada servicio puede escalarse horizontalmente según sus necesidades específicas (ej. escalar más instancias del servicio de adquisición de datos si hay muchos usuarios).
    \item \textbf{Flexibilidad Tecnológica:} Cada servicio puede desarrollarse con la tecnología más adecuada para su tarea específica (diferentes lenguajes, bases de datos).
    \item \textbf{Despliegue Independiente:} Los cambios en un servicio pueden desplegarse sin necesidad de redesplegar todo el sistema, agilizando las actualizaciones.
\end{itemize}

\subsection{Bases de Datos de Series Temporales}
\label{subsec:ea_db_timeseries}

Los datos generados por dispositivos wearables como Fitbit\textsuperscript{\textregistered} son inherentemente datos de series temporales: secuencias de mediciones indexadas por tiempo (timestamp). Si bien es posible almacenar estos datos en bases de datos relacionales tradicionales (como PostgreSQL o MySQL), las bases de datos especializadas en series temporales (TSDB - Time Series Databases) están optimizadas para este tipo de carga de trabajo \cite{dbengines_timeseries_ranking}.

Las TSDB suelen ofrecer ventajas significativas para datos de series temporales, como:
\begin{itemize}
    \item \textbf{Alto Rendimiento en Ingesta:} Optimizadas para escribir grandes volúmenes de datos nuevos secuencialmente en el tiempo.
    \item \textbf{Consultas Eficientes Basadas en Tiempo:} Indexación y funciones específicas para agregar, muestrear o filtrar datos por rangos de tiempo de forma muy rápida.
    \item \textbf{Compresión de Datos:} Técnicas específicas para comprimir datos temporales, que suelen tener cierta redundancia o patrones, ahorrando espacio de almacenamiento.
    \item \textbf{Políticas de Retención de Datos:} Facilidades para descartar automáticamente datos antiguos que ya no son necesarios (ej. mantener datos con granularidad de minutos por 1 mes, pero solo resúmenes diarios después de eso).
\end{itemize}
Ejemplos populares de TSDB incluyen InfluxDB y TimescaleDB (una extensión para PostgreSQL) \cite{influxdb_docs, timescaledb_docs}. Para este proyecto, se optó por \textbf{TimescaleDB} debido a su integración nativa con PostgreSQL, lo que permite combinar las ventajas de una TSDB con las capacidades de una base de datos relacional robusta, su uso de SQL estándar para las consultas y su madurez como proyecto \cite{timescaledb_docs}.

\subsection{Herramientas de Backend y Procesamiento}
\label{subsec:ea_backend_tools}

El backend del sistema, responsable de orquestar la autenticación, la adquisición de datos, el procesamiento y la exposición de APIs internas o para el frontend, se ha desarrollado utilizando \textbf{Python}. Python es una elección popular para el desarrollo web y el procesamiento de datos debido a su sintaxis clara, su amplio ecosistema de librerías y su gran comunidad \cite{python_website}.

Como framework web para construir las APIs de los microservicios, se ha empleado \textbf{Flask} \cite{flask_docs}, un microframework ligero que se alinea con la filosofía de microservicios al mantener cada componente simple y modular.

Para la adquisición periódica de datos desde la API de Fitbit\textsuperscript{\textregistered} (una tarea que debe ejecutarse de forma programada en segundo plano para cada usuario vinculado), se utiliza \textbf{crontab}, el planificador de tareas estándar de sistemas Unix/Linux. Crontab permite definir trabajos (jobs) que se ejecutan según expresiones cron, siendo una solución robusta y probada para la ejecución programada de scripts.

\subsection{Tecnologías de Frontend/Visualización}
\label{subsec:ea_frontend_viz}

Para presentar la información monitorizada de forma clara y útil a los cuidadores, se ha desarrollado un panel web sencillo e intuitivo. Dada la base tecnológica en Python/Flask y la naturaleza de los datos (series temporales, gráficos estadísticos), la interfaz se construye con plantillas HTML, CSS y JavaScript integradas en Flask. La visualización se realiza mediante librerías JavaScript como Chart.js, permitiendo mostrar líneas de tiempo, resúmenes e indicadores clave de manera interactiva y comprensible. Esta solución prioriza la simplicidad y la mantenibilidad.

\section{Consideraciones Éticas y Legales (RGPD)}
\label{sec:ea_rgpd}

El sistema implementa los principios fundamentales del RGPD: consentimiento explícito mediante OAuth 2.0, uso limitado y minimizado de los datos, exactitud y conservación adecuada, y medidas técnicas y organizativas para garantizar la seguridad y confidencialidad (como HTTPS, almacenamiento seguro de credenciales y control de accesos). Se facilita el ejercicio de los derechos de los usuarios (acceso, rectificación, supresión, etc.) mediante mecanismos accesibles en la propia aplicación. La documentación de políticas y registros de consentimiento permite demostrar el cumplimiento normativo.
% -*- coding: utf-8 -*-
\chapter{Análisis y Metodología}
\label{chap:requisitos_metodologia}

Este capítulo detalla los requisitos que debe cumplir el sistema desarrollado y la metodología seguida durante su construcción. Los requisitos se dividen en funcionales, que describen las capacidades del sistema, y no funcionales, que especifican sus atributos de calidad.

\section{Requisitos Funcionales}
\label{sec:requisitos_funcionales}

Los requisitos funcionales (RF) definen las tareas y servicios específicos que el sistema de monitorización debe ser capaz de realizar. Han sido identificados a partir de los objetivos del proyecto definidos en el Capítulo \ref{chap:introduccion} y las necesidades del escenario de uso previsto (personal autorizado monitorizando a residentes/ancianos). A continuación, se enumeran los requisitos funcionales clave implementados en el prototipo:

\begin{description}
    \item[RF-01: Autenticación de Personal] El sistema proporciona un mecanismo de inicio de sesión para el personal autorizado, utilizando credenciales compartidas (usuario/contraseña) gestionadas mediante variables de entorno. Permite también cerrar la sesión de forma segura.
    \item[RF-02: Gestión de Vinculaciones Fitbit\textsuperscript{\textregistered}-Nombre] El sistema permite al personal autorizado:
        \begin{itemize}
            \item Visualizar la lista de cuentas de email de Fitbit\textsuperscript{\textregistered} disponibles para vincular (obtenidas de la base de datos).
            \item Asociar un nombre identificativo a una cuenta de email de Fitbit\textsuperscript{\textregistered} durante el proceso de vinculación inicial o al reasignar un dispositivo/email ya existente.
            \item Visualizar la lista de cuentas actualmente vinculadas, mostrando el nombre asociado y el email de Fitbit\textsuperscript{\textregistered}. 
        \end{itemize}
        Cada email de Fitbit se asocia a un único nombre en cada momento.
    \item[RF-03: Vinculación de Cuentas Fitbit\textsuperscript{\textregistered} (OAuth 2.0)] Tras seleccionar un email y asociarle un nombre, el sistema gestiona de forma transparente y segura el flujo de autorización OAuth 2.0 (\textit{Authorization Code Grant}) con Fitbit\textsuperscript{\textregistered}, redirigiendo al usuario a Fitbit para la autenticación y autorización de permisos, y manejando el callback para obtener los tokens.
    \item[RF-04: Gestión de Reasignación] El sistema permite reasignar una cuenta de email de Fitbit\textsuperscript{\textregistered} a un nuevo nombre identificativo, gestionando la reautorización si los tokens no son válidos.
    \item[RF-05: Adquisición Automática de Datos] El sistema obtiene periódicamente los datos de salud y actividad disponibles (frecuencia cardíaca, patrones de sueño, pasos) de todas las cuentas Fitbit\textsuperscript{\textregistered} activamente vinculadas y con tokens válidos, mediante scripts programados con \textbf{cron}.
    \item[RF-06: Almacenamiento de Datos Temporales] El sistema persiste de forma estructurada los datos adquiridos de Fitbit\textsuperscript{\textregistered} en la base de datos de series temporales (TimescaleDB), asegurando que cada dato quede asociado al email y nombre correspondientes y conserve su información temporal (timestamp).
    \item[RF-07: Procesamiento Básico de Datos] El sistema realiza un procesamiento mínimo sobre los datos crudos recibidos de la API antes de su almacenamiento o visualización, como la validación de formato y cálculo del tiempo total de sueño, extracción de pasos totales diarios, etc.
    \item[RF-08: Visualización de Datos Históricos] El sistema ofrece un panel de visualización (dashboard) accesible vía web para el personal autorizado. Este panel permite seleccionar un residente (por su nombre/email asociado) y mostrar de forma clara e intuitiva sus datos históricos (frecuencia cardíaca, sueño, actividad) mediante gráficos interactivos y tablas resumen.
    \item[RF-09: Gestión Segura de Tokens] El sistema implementa mecanismos seguros para el almacenamiento y la gestión del ciclo de vida (obtención, uso, refresco, manejo de errores en la revocación) de los tokens de acceso y refresco de OAuth 2.0 obtenidos de Fitbit\textsuperscript{\textregistered}. Los tokens se almacenan cifrados en la base de datos.
    \item[RF-10: Evaluación de Criterios de Alerta] El sistema evalúa periódicamente los datos almacenados (actividad, sueño, FC) contra un conjunto predefinido de criterios y umbrales para identificar posibles situaciones de alerta.
    \item[RF-11: Generación y Registro de Alertas] Ante la detección de una condición de alerta según los criterios definidos, el sistema registra dicho evento y lo muestra en el dashboard para su revisión por el personal autorizado.
    \item[RF-12: Priorización y Contextualización de Alertas] El sistema asigna niveles de prioridad (bajo, medio, alto) a las alertas basándose en la magnitud de la desviación respecto al umbral y compara los valores actuales con la línea base reciente del propio usuario.
\end{description}

Estos requisitos funcionales constituyen la base sobre la cual se ha diseñado e implementado la funcionalidad del prototipo actual.

% -*- coding: utf-8 -*-

% --- Resto del Capítulo 3 ---

\section{Requisitos No Funcionales}
\label{sec:requisitos_no_funcionales}

Además de las funciones que debe realizar, el sistema debe cumplir ciertos atributos de calidad y restricciones operativas, conocidos como Requisitos No Funcionales (RNF). Estos requisitos definen \textit{cómo} debe operar el sistema. Para este proyecto, se han considerado los siguientes RNF clave:

\begin{description}
    \item[RNF-01: Usabilidad (Interfaz de Personal)] La interfaz web destinada al personal autorizado debe ser intuitiva y fácil de usar, especialmente en las tareas críticas como la vinculación/reasignación de dispositivos y la visualización de datos (cuando esté implementada). Los mensajes de error deben ser claros y orientativos.
    \item[RNF-02: Rendimiento (Adquisición y Almacenamiento)] El proceso de adquisición de datos, tanto diario como intradía, se ejecuta mediante scripts programados con cron, respetando los límites de la API de Fitbit\textsuperscript{\textregistered} y procesando los datos de múltiples usuarios de forma eficiente. La escritura en la base de datos PostgreSQL/TimescaleDB es adecuada para el volumen de datos manejado en el prototipo. El dashboard web permite una consulta fluida de los datos históricos.
    \item[RNF-03: Seguridad]
        \begin{itemize}
            \item Autenticación: El acceso a la aplicación web está protegido mediante autenticación compartida.
            \item Autorización: Solo el personal autenticado puede realizar acciones o ver datos.
            \item Gestión de Tokens: Los tokens OAuth 2.0 se almacenan cifrados en la base de datos y se transmiten de forma segura.
            \item Comunicaciones: Toda la comunicación sensible se realiza sobre HTTPS en entornos de producción.
            \item Protección Web: El sistema aplica buenas prácticas de seguridad web para prevenir vulnerabilidades comunes, siguiendo recomendaciones como las del OWASP Top 10 \cite{owasp_top10}.
        \end{itemize}
    \item[RNF-04: Fiabilidad y Disponibilidad] El sistema debe ser razonablemente fiable. La ejecución programada de los scripts de adquisición mediante `cron` debe ser robusta. Los scripts y la aplicación web deben manejar correctamente errores esperables (ej. fallos de red, errores de la API Fitbit, errores de BD) registrando la información relevante para diagnóstico sin detener por completo el servicio.
    \item[RNF-05: Mantenibilidad] El código fuente está organizado en módulos Python (`app.py`, `auth.py`, `db.py`, `fitbit.py`, etc.), es legible y está comentado. Se utiliza el sistema de control de versiones Git, con el repositorio alojado en GitHub \cite{github_repo_proyecto}, para gestionar los cambios y facilitar la colaboración o futuras revisiones.
    \item[RNF-06: Escalabilidad (Diseño)] La arquitectura (aplicación Flask modular, scripts independientes de adquisición, base de datos PostgreSQL/TimescaleDB) proporciona una base que podría escalarse (ej. ejecutando más instancias de los scripts de adquisición, escalando la base de datos) si fuera necesario manejar un mayor volumen de usuarios o datos en el futuro.
    \item[RNF-07: Privacidad (Cumplimiento RGPD)] El sistema se ha diseñado siguiendo los principios del RGPD, incluyendo el cifrado de tokens y la gestión del consentimiento mediante el flujo OAuth.
    \item[RNF-08:] \textbf{Rendimiento y Fiabilidad de Alertas:} La evaluación de los criterios de alerta se realiza de forma eficiente y la lógica de detección es robusta frente a datos faltantes o erróneos, minimizando falsos positivos dentro de lo posible con los criterios definidos.
    \item[RNF-09:] \textbf{Latencia de Alertas:} El tiempo entre la disponibilidad del dato relevante y la generación/registro de la alerta debe ser suficientemente bajo para permitir una respuesta oportuna (ej., dentro de X horas para alertas diarias, Y minutos para intradía).
    \item[RNF-10:] \textbf{Precisión de Alertas:} El sistema debe diseñarse para minimizar las falsas alarmas (falsos positivos) y la omisión de eventos reales (falsos negativos), aunque se reconoce el compromiso inherente (trade-off) entre sensibilidad y especificidad con los sistemas basados en reglas.
\end{description}

\textit{(Nota: La sección 3.3 sobre Casos de Uso se omite en esta versión para mayor brevedad).}

\section{Metodología de Desarrollo}
\label{sec:metodologia}

El desarrollo de este Trabajo Fin de Grado se ha abordado siguiendo un enfoque principalmente \textbf{iterativo e incremental}, adaptado a la naturaleza exploratoria y de creación de prototipos propia de un proyecto académico de este tipo. No se siguió estrictamente una metodología ágil formal como Scrum, pero se adoptaron algunos de sus principios, como la flexibilidad ante cambios y la entrega de valor funcional en ciclos cortos.

Las fases principales del desarrollo se pueden resumir en:

\begin{enumerate}
    \item \textbf{Investigación y Definición (Fase Inicial):}
        \begin{itemize}
            \item Revisión bibliográfica sobre monitorización remota, wearables y tecnologías relevantes.
            \item Estudio detallado de la documentación de la API de Fitbit\textsuperscript{\textregistered} y el protocolo OAuth 2.0 con PKCE.
            \item Definición inicial de los objetivos y alcance del proyecto en colaboración con el tutor.
            \item Identificación de los requisitos funcionales y no funcionales preliminares.
        \end{itemize}
    \item \textbf{Diseño de la Arquitectura y Tecnologías:}
        \begin{itemize}
            \item Toma de decisiones sobre la arquitectura general: aplicación web Flask, scripts Python independientes para adquisición de datos, base de datos PostgreSQL con TimescaleDB para métricas, y programación de tareas periódicas mediante \textbf{cron}.
            \item Selección de las tecnologías principales (Python, Flask, psycopg2, cryptography, etc.).
            \item Diseño del esquema de la base de datos (tabla `users` y estructura pensada para tablas de métricas) y las interfaces entre componentes (rutas Flask, funciones en módulos Python).
        \end{itemize}
    \item \textbf{Implementación Iterativa (Ciclos de Desarrollo):}
        \begin{itemize}
            \item Desarrollo incremental de la funcionalidad principal, priorizando los módulos clave:
            \item Implementación del flujo de autenticación OAuth 2.0 con Fitbit\textsuperscript{\textregistered} (`auth.py`).
            \item Desarrollo del módulo de base de datos (`db.py`) incluyendo cifrado de tokens (`encryption.py`).
            \item Creación de la aplicación web Flask (`app.py`) con las rutas para la gestión de vinculaciones y autenticación del personal.
            \item Desarrollo de los scripts independientes para la adquisición de datos diarios (`fitbit.py`) e intradía (`fitbit\_intraday.py`).
            \item Configuración de la ejecución programada mediante \textbf{cron} y scripts `.sh`.
            \item Tratamiento de los datos y generación de alertas.
            \item Implementación de la interfaz de visualización (dashboard web con Flask y plantillas HTML/JS).
            \item Realización de pruebas funcionales manuales y depuración durante el desarrollo.
        \end{itemize}
    \item \textbf{Pruebas y Validación:}
        \begin{itemize}
            \item Ejecución de pruebas sobre el prototipo desplegado (en VM) para verificar el cumplimiento de los requisitos implementados (vinculación, adquisición, almacenamiento básico).
            \item Pruebas del flujo completo de vinculación y adquisición programada.
            \item Depuración y corrección de errores encontrados.
        \end{itemize}
    \item \textbf{Documentación:}
        \begin{itemize}
            \item Redacción de la memoria del TFG (este documento).
            \item Comentarios en el código fuente.
            \item Elaboración de diagramas y esquemas necesarios.
        \end{itemize}
\end{enumerate}

Para la gestión del código fuente y el control de versiones se utilizó \textbf{Git}, alojando el repositorio centralizado en la plataforma \textbf{GitHub} \cite{github_repo_proyecto}, lo que permitió un seguimiento detallado de los cambios y la posibilidad de colaboración. La gestión de tareas se realizó mediante seguimiento personal y comunicación con el tutor.

\section{Herramientas de Apoyo a la Redacción}
\label{sec:apoyo_redaccion}

La redacción de este TFG ha requerido un esfuerzo significativo para garantizar la claridad, coherencia y precisión técnica. Para optimizar el proceso, se empleó una estrategia mixta: la generación inicial de contenido se realizó mediante dictado por voz y escritura directa en procesadores de texto, priorizando la captura ágil de ideas y descripciones técnicas.

Posteriormente, se utilizó asistencia de modelos de lenguaje avanzados, en particular Gemini (Google), como herramienta de apoyo para:
\begin{itemize}
    \item Mejorar la claridad y concisión de las frases.
    \item Corregir errores gramaticales y de estilo.
    \item Mantener un tono formal y académico homogéneo.
    \item Optimizar la estructura y el flujo de los párrafos.
\end{itemize}
El uso de Gemini se limitó a la reformulación y revisión lingüística, bajo directrices precisas, sin delegar la responsabilidad sobre el contenido técnico, la estructura ni la validación final, que son íntegramente del autor. Esta asistencia permitió agilizar el pulido del texto y elevar la calidad formal de la memoria, sin comprometer el rigor ni la autoría intelectual.

    % -*- coding: utf-8 -*-
% --- Capítulo 4 ---

\chapter{Diseño y Arquitectura del Sistema}
\label{chap:diseno_arquitectura} 

Este capítulo describe la arquitectura global del sistema y las decisiones de diseño clave que guían su implementación. Los detalles técnicos completos de implementación se desarrollan en el Capítulo \ref{chap:implementacion}.

\section{Arquitectura General}
\label{sec:arquitectura_general}

El sistema sigue una arquitectura de microservicios, con un backend Flask que interactúa con una base de datos PostgreSQL/TimescaleDB y la API externa de Fitbit\textsuperscript{\textregistered}. La adquisición de datos se realiza mediante scripts Python independientes ejecutados por el planificador del sistema (`cron`). La Figura~\ref{fig:arquitectura_general} ilustra esta arquitectura.

\begin{figure}[htbp] 
    \centering
    \includegraphics[width=0.9\textwidth]{imagenes/arquitectura_general.png}
    \caption{Arquitectura General del Sistema de Monitorización.}
    \label{fig:arquitectura_general}
\end{figure}

Los componentes principales y sus responsabilidades son:

\begin{itemize}
    \item \textbf{Cliente Web:} Interfaz HTML/CSS/JavaScript con visualizaciones Chart.js y soporte multilingüe.
    \item \textbf{Backend Flask:} Gestiona autenticación, vinculación de dispositivos y sirve el dashboard.
    \item \textbf{Scripts de Adquisición:} Procesos independientes para obtener datos de Fitbit.
    \item \textbf{Base de Datos:} PostgreSQL con TimescaleDB para series temporales.
\end{itemize}

\section{Diseño del Pipeline de Datos}
\label{sec:diseno_pipeline}

El pipeline de datos del sistema está diseñado para garantizar la adquisición, validación, almacenamiento y análisis eficiente de la información proveniente de los dispositivos Fitbit. El flujo se estructura en cinco componentes principales, como se ilustra en la Figura~\ref{fig:flujo_datos}:

\begin{enumerate}
    \item \textbf{Orquestación:} Scripts Python (\texttt{fitbit.py}, \texttt{fitbit\_intraday.py}) ejecutados periódicamente por \texttt{cron} para la adquisición automática de datos.
    \item \textbf{Adquisición:} Obtención de datos de la API de Fitbit mediante tokens OAuth 2.0 cifrados, con manejo robusto de errores y refresco automático de credenciales.
    \item \textbf{Almacenamiento:} Persistencia en hipertablas TimescaleDB optimizadas para series temporales (\texttt{daily\_summaries}, \texttt{intraday\_metrics}, \texttt{sleep\_logs}).
    \item \textbf{Procesamiento:} Evaluación automática de reglas de alerta basadas en umbrales científicos y patrones individuales, generando alertas clínicas cuando corresponde.
    \item \textbf{Visualización:} Interfaz web con dashboards interactivos para consulta, filtrado y exportación de datos y alertas.
\end{enumerate}

\begin{figure}[htbp]
    \centering
    \includegraphics[width=1\textwidth, height=0.3\textheight]{imagenes/flujo_datos.png}
    \caption{Flujo de datos desde la API de Fitbit hasta la visualización en el dashboard.}
    \label{fig:flujo_datos}
\end{figure}

Este diseño modular desacopla la adquisición del análisis y la visualización, facilitando la escalabilidad y la integración futura de nuevas fuentes de datos o reglas de alerta. La evaluación de alertas se realiza de forma automática tras cada ingesta, comparando los datos recientes con líneas base individuales y umbrales predefinidos, almacenando los resultados en la tabla \texttt{alerts} para su posterior revisión clínica.(ver implementación en la Sección~\ref{subsec:anexo_actividad_drop}).

\subsection{Evaluación Dinámica de Reglas de Alerta}
La detección de anomalías y generación de alertas se realiza de forma automática tras cada ingesta de datos, mediante funciones desacopladas del frontend y definidas en un módulo específico. Las reglas pueden consultar ventanas temporales (por ejemplo, 7 días de pasos o sueño) y comparar los valores actuales con medias, umbrales o desviaciones estándar. El resultado se almacena en la tabla \texttt{alerts}, permitiendo su posterior revisión clínica.

El diseño modular facilita la incorporación de nuevas reglas, la extensión a patrones más complejos o incluso la integración futura de algoritmos de aprendizaje automático.


\section{Diseño del Backend (Aplicación Flask y Scripts)}
\label{sec:diseno_backend}

El backend se estructura en dos componentes principales:

\begin{enumerate}
    \item \textbf{Aplicación Web Flask:} 
        \begin{itemize}
            \item Servir las páginas HTML para el login, la selección de email, la asignación de nombre y las confirmaciones (usando plantillas Jinja2 multilenguaje).
            \item Gestionar el flujo OAuth 2.0: generar parámetros (`state`, `code\_challenge`), construir la URL de autorización, manejar la redirección del usuario a Fitbit\textsuperscript{\textregistered} y procesar el `callback`.
            \item Interactuar con \texttt{auth.py} para obtener los tokens a partir del código de autorización.
            \item Interactuar con \texttt{db.py} y \texttt{encryption.py} para guardar/actualizar la información del usuario y los tokens cifrados en la tabla \texttt{users}.
            \item Servir el dashboard y exponer endpoints API para la precarga y actualización eficiente de datos, optimizando la latencia y la escalabilidad. La implementación de las rutas principales se detalla en la Sección~\ref{subsec:anexo_rutas} del Anexo.
            \item Gestionar logs y errores de forma centralizada para garantizar la robustez del sistema.
        \end{itemize}
    \item \textbf{Scripts de Adquisición:} 
        \begin{itemize}
            \item Obtiene y descifra los tokens (\texttt{encryption.py}, \texttt{db.py}).
            \item Verifica la validez del token de acceso (\texttt{expires\_at}). Si es necesario, intenta refrescarlo usando el token de refresco (\texttt{auth.py}, \texttt{fitbit.py}) y actualiza los tokens cifrados y la expiración en la BD (\texttt{db.py}).
            \item Si los tokens son válidos, realiza las llamadas correspondientes a la API de Fitbit\textsuperscript{\textregistered} (\texttt{fitbit.py}, \texttt{fitbit\_intraday.py}).
            \item Procesa la respuesta JSON y maneja posibles errores, registrando los fallos y continuando con el siguiente usuario para garantizar la robustez.
            \item Se conecta a la BD (\texttt{db.py}) para insertar los datos procesados en las tablas/hipertablas de TimescaleDB apropiadas.
            \item Tras la inserción, ejecuta la lógica de evaluación de alertas de forma desacoplada, permitiendo la extensión futura a reglas más complejas o nuevos tipos de datos.
        \end{itemize}
\end{enumerate}
Esta separación y modularidad permiten que la adquisición de datos no bloquee la aplicación web, facilitan la escalabilidad (añadiendo más scripts o fuentes de datos) y mejoran la mantenibilidad del sistema.

\section{Diseño de la Base de Datos (PostgreSQL + TimescaleDB)}
\label{sec:diseno_bd}

La base de datos combina PostgreSQL con la extensión TimescaleDB para optimizar el manejo de series temporales. La Figura~\ref{fig:esquema_relacional} muestra el esquema completo.

\begin{figure}[htbp]
    \centering
    \includegraphics[width=0.9\textwidth]{imagenes/esquema_relacional.png}
    \caption{Esquema relacional de la base de datos del sistema.}
    \label{fig:esquema_relacional}
\end{figure}

El modelo se estructura en:
\begin{itemize}
    \item \textbf{Tabla Relacional:} \texttt{users} para gestión de usuarios y tokens
    \item \textbf{Hipertablas TimescaleDB:} Para datos temporales (métricas diarias, intradía, sueño y alertas)
\end{itemize}

Las sentencias SQL detalladas y la descripción de campos se encuentran en el Anexo~\ref{app:db_schema}.


\section{Diseño de la Interfaz de Usuario}
\label{sec:diseno_ui}

La interfaz de usuario (UI) del sistema está diseñada para ser intuitiva, accesible y eficiente con soporte multilingüe (español/inglés) facilitando tanto la gestión administrativa como la visualización clínica de los datos monitorizados. Se compone de dos grandes bloques:

\begin{itemize}
    \item \textbf{Interfaz de Gestión y Vinculación (Flask/HTML + Bootstrap):}
        \begin{itemize}
            \item Login y gestión de dispositivos
            \item Flujo OAuth 2.0 para vinculación
            \item Soporte multilingüe (ES/EN)
        \end{itemize}
    \item \textbf{Dashboard de Visualización y Alertas:}
        \begin{itemize}
            \item Implementado como un conjunto de vistas Flask con plantillas Jinja2 y componentes interactivos (JavaScript, Chart.js).
            \item Permite visualizar resúmenes diarios, métricas intradía, patrones de sueño y alertas recientes para cada usuario.
            \item Incluye filtros avanzados (por usuario, fecha, tipo de alerta, prioridad, etc.) y opciones de exportación a CSV.
            \item La precarga de datos y la optimización de consultas mejoran la velocidad de carga y la experiencia de usuario, especialmente en el dashboard de alertas. La implementación de las llamadas AJAX para la actualización dinámica se muestra en la Sección~\ref{subsec:anexo_formularios} del Anexo.
        \end{itemize}
\end{itemize}

El diseño modular y el uso de tecnologías estándar (Flask, Bootstrap, Chart.js) facilitan el mantenimiento y la extensión del sistema, mientras que la internacionalización integrada mejora la accesibilidad para usuarios de diferentes regiones.

\subsection{Ficha de Usuario y Visualización Individual}
Una de las vistas más relevantes del sistema es la \textbf{ficha de usuario} (\texttt{user\_detail.html}), que centraliza toda la información relevante de cada paciente o usuario monitorizado. Esta vista está diseñada para facilitar la toma de decisiones clínicas y el seguimiento personalizado, integrando:

\begin{itemize}
    \item \textbf{Resumen de datos personales:} nombre, email, fecha de registro y estado de actividad reciente.
    \item \textbf{Métricas clave del día:} pasos, frecuencia cardíaca, horas de sueño, calorías, etc., resaltando visualmente cualquier valor anómalo o alerta activa.
    \item \textbf{Alertas recientes:} listado de alertas generadas en los últimos días, con posibilidad de reconocerlas directamente desde la ficha.
    \item \textbf{Gráficos interactivos:} evolución semanal de pasos, sueño, frecuencia cardíaca y minutos activos, así como visualización intradía y análisis de patrones de inactividad, implementados con Chart.js.
    \item \textbf{Tabs de navegación:} acceso rápido a diferentes vistas (resumen diario, intradía, semanal, alertas, inactividad).
    \item \textbf{Exportación y actualización:} botones para exportar datos a CSV y actualizar la información en tiempo real.
    \item \textbf{Accesibilidad y usabilidad:} diseño responsivo, iconografía clara, leyendas de colores para priorización y mensajes de estado.
\end{itemize}

Esta ficha ejemplifica la integración de todos los módulos del sistema (adquisición, almacenamiento, análisis y visualización), permitiendo al personal sanitario o gestor acceder de forma rápida y comprensible a la información más relevante para cada usuario.

\subsection{Diseño del Módulo de Alertas}
\label{subsec:diseno_alertas}

El módulo de alertas es un componente central del sistema, encargado de analizar los datos almacenados y detectar automáticamente situaciones clínicas relevantes o anomalías en la actividad, el sueño o la frecuencia cardíaca de los usuarios. Su diseño es modular y desacoplado del frontend, permitiendo su ejecución periódica tras cada ingesta de datos y facilitando la extensión futura con nuevas reglas o métricas.

\begin{itemize}
    \item \textbf{Acceso a datos históricos:} El módulo dispone de funciones específicas (en \texttt{db.py}) para recuperar las métricas necesarias en ventanas temporales (por ejemplo, los últimos 7 días de pasos o sueño) y realizar comparaciones con la línea base individual de cada usuario.
    \item \textbf{Lógica de comparación y reglas:} Las reglas de alerta están implementadas en Python (\texttt{alert\_rules.py}) y se basan en umbrales científicos, porcentajes de cambio, desviaciones estándar o rangos fisiológicos. Cada función evalúa si los datos actuales superan los límites definidos y, en caso afirmativo, genera una alerta con prioridad, tipo, valor disparador y detalles clínicos.
    \item \textbf{Registro y gestión de alertas:} Las alertas detectadas se almacenan en la tabla \texttt{alerts}, asociando cada evento con el usuario, la métrica, la prioridad y una descripción. Esto permite su posterior visualización, filtrado y exportación desde el dashboard.
    \item \textbf{Extensibilidad:} El diseño permite añadir fácilmente nuevas reglas, métricas o fuentes de datos, así como adaptar los umbrales según la evidencia clínica o la experiencia práctica.
\end{itemize}

\subsubsection{Manejo de Datos Faltantes o Erróneos}
La calidad de los datos de wearables puede verse afectada por desconexiones, falta de uso o errores de medición. El sistema implementa estrategias robustas para minimizar falsas alarmas y garantizar la fiabilidad de las alertas:
\begin{itemize}
    \item Se requiere un porcentaje mínimo de días con datos válidos en las ventanas temporales para evaluar una alerta (por ejemplo, al menos 5 de 7 días).
    \item Se aplican filtros de rango fisiológico antes de procesar los datos (por ejemplo, descartar valores de frecuencia cardíaca fuera de 30-200 bpm o pasos diarios superiores a 50.000).
    \item Los datos faltantes críticos generan alertas de calidad de datos, permitiendo al personal identificar posibles problemas de uso o sincronización.
\end{itemize}

\subsubsection{Optimización y Rendimiento}
Dado el volumen potencial de datos y la necesidad de evaluaciones históricas frecuentes, el sistema optimiza el acceso y procesamiento mediante:
\begin{itemize}
    \item Índices compuestos en las hipertablas de TimescaleDB (por ejemplo, sobre \texttt{(user\_id, time)}) para acelerar las consultas.
    \item Cálculos eficientes en los scripts, reutilizando los datos recuperados para varias métricas cuando es posible.
    \item Procesamiento asíncrono y desacoplado mediante la ejecución periódica por \texttt{cron}, evitando que la evaluación de alertas afecte la experiencia de usuario en la interfaz web.
\end{itemize}

\subsubsection{Diagrama de Flujo del Proceso de Detección de Alertas}
El proceso lógico para la detección y registro de alertas sigue el flujo ilustrado en la Figura~\ref{fig:diagrama_alertas}:

\begin{figure}[htbp]
    \centering
    \includegraphics[width=0.9\textwidth,height=0.6\textheight]{imagenes/diagrama_alertas.png} 
    \caption{Diagrama de flujo del proceso de detección y priorización de alertas.}
    \label{fig:diagrama_alertas}
\end{figure}

Este diseño permite que la evaluación de alertas se beneficie de las optimizaciones de consulta de TimescaleDB y se mantenga desacoplada de la interfaz de usuario, garantizando robustez, escalabilidad y relevancia clínica.
Los umbrales detallados y su justificación se documentan en el Anexo~\ref{anexo:anexo_umbrales}.

\subsection{Arquitectura de la Interfaz Web y Dashboards}
\label{sec:arquitectura_dashboard}

La interfaz web del sistema está compuesta por dos dashboards principales:

\begin{itemize}
    \item \textbf{Dashboard de Alertas:} Permite al personal autorizado visualizar, filtrar y exportar todas las alertas generadas por el sistema. Incluye filtros por fecha, usuario, tipo de alerta, prioridad y estado de reconocimiento. Cada alerta puede ser reconocida manualmente y se muestra información detallada, incluyendo datos intradía relevantes y contexto clínico.
    \item \textbf{Dashboard de Usuarios:} Presenta un listado de todos los usuarios monitorizados, con búsqueda por nombre o email y estado de actividad reciente. Desde aquí se accede a la ficha de usuario.
\end{itemize}

La \textbf{ficha de usuario} incluye:
\begin{itemize}
    \item Resumen diario de métricas clave (pasos, frecuencia cardíaca, sueño, calorías, etc.).
    \item Visualización de datos intradía (gráficos de pasos, FC, calorías, minutos activos).
    \item Resumen semanal (tendencias de pasos, sueño, actividad, etc.).
    \item Listado de alertas recientes y posibilidad de exportarlas.
    \item Análisis de patrones de inactividad (detección de periodos prolongados sin actividad).
    \item Exportación de datos históricos e intradía en formato CSV.
\end{itemize}

La navegación entre dashboards y fichas de usuario es intuitiva y está protegida por autenticación. El diseño prioriza la claridad visual y la accesibilidad para facilitar la toma de decisiones clínicas.

% !TEX root = ../main.tex % Indica a algunos editores cuál es el fichero raíz
\chapter{Implementación}
\label{chap:implementacion}

Este capítulo detalla el proceso de construcción del sistema de monitorización remota, materializando el diseño arquitectónico expuesto en el Capítulo \ref{chap:diseno_arquitectura} en componentes software funcionales. Se aborda la configuración del entorno, la implementación de los componentes del backend (aplicación web y scripts de adquisición), la configuración y manejo de la base de datos, la lógica de procesamiento de datos, las medidas de seguridad aplicadas y la interfaz de visualización. Finalmente, se comentan algunos de los desafíos técnicos encontrados. El código fuente completo está disponible en el repositorio del proyecto \citep{github_repo_proyecto}.

\section{Entorno de Desarrollo y Tecnologías}
\label{sec:impl_entorno}

La implementación del prototipo se ha realizado utilizando un conjunto de tecnologías seleccionadas por su adecuación a los requisitos del proyecto y su robustez en entornos de producción. El lenguaje de programación principal es \textbf{Python} (versión 3.x) \citep{python_website}, aprovechando su flexibilidad y la riqueza de su ecosistema para desarrollo web, integración de APIs y procesamiento de datos.

Las herramientas y librerías clave empleadas son:
\begin{itemize}
    \item \textbf{Control de Versiones:} \textbf{Git} y \textbf{GitHub} \citep{github_repo_proyecto} para el control de versiones
    \item \textbf{Backend Web:} \textbf{Flask} \citep{flask_docs} como microframework para construir la aplicación web principal (\texttt{app.py}), gestionando rutas, peticiones y respuestas HTTP. Se complementa con \textbf{Flask-Login} para la gestión de sesiones y autenticación del personal.
    \item \textbf{Base de Datos:} \textbf{PostgreSQL} \citep{postgresql_docs} como sistema gestor de base de datos relacional, extendido con \textbf{TimescaleDB} \citep{timescaledb_docs} para la gestión eficiente de datos de series temporales. La interacción desde Python se realiza mediante la librería \textbf{psycopg2}.
    \item \textbf{Seguridad:} \textbf{cryptography} \citep{cryptography_docs} se emplea para el cifrado simétrico (Fernet) de los tokens OAuth 2.0 almacenados en la base de datos. Las claves y credenciales sensibles se gestionan mediante variables de entorno, nunca en el código fuente.
    \item \textbf{Comunicaciones API:} \textbf{requests} \citep{requests_docs} para realizar las llamadas HTTP a la API web de Fitbit.
    \item \textbf{Visualización:} La interfaz de usuario y el dashboard se implementan con \textbf{Flask}, plantillas \textbf{Jinja2}, \textbf{Bootstrap} para el diseño responsivo y \textbf{Chart.js} para la generación de gráficos interactivos en el navegador. Esta pila permite una integración directa con el backend y una experiencia de usuario moderna y eficiente.
    \item \textbf{Planificación de Tareas:} El sistema \textbf{cron} del sistema operativo se utiliza para lanzar los scripts de adquisición de datos (\texttt{fitbit.py}, \texttt{fitbit\_intraday.py}) a intervalos regulares, mediante scripts wrapper.
\end{itemize}

La elección de estas tecnologías permite una arquitectura modular, fácilmente mantenible y escalable, y facilita el despliegue en servidores Linux estándar. Todas las dependencias son de código abierto y ampliamente utilizadas en la industria, lo que garantiza soporte y seguridad a largo plazo.

\section{Implementación del Backend}
\label{sec:impl_backend}

El backend del sistema se compone de dos bloques principales: la aplicación web desarrollada con Flask y los scripts independientes de adquisición de datos. Esta separación permite una arquitectura modular, robusta y fácilmente extensible.

\subsection{Aplicación Web Flask (\texttt{app.py})}
La aplicación Flask centraliza las siguientes funcionalidades:
\begin{itemize}
    \item \textbf{Autenticación del Personal:} Gestiona el inicio y cierre de sesión del personal autorizado utilizando Flask-Login y credenciales almacenadas de forma segura (variables de entorno).
    \item \textbf{Gestión de Vinculaciones:} Proporciona rutas y plantillas HTML (Jinja2) para visualizar cuentas de Fitbit, asociar nombres a emails y gestionar vinculaciones activas.
    \item \textbf{Orquestación del Flujo OAuth 2.0:} Inicia el proceso de autorización con Fitbit, gestiona el intercambio de códigos por tokens mediante el módulo \texttt{auth.py}, y almacena los tokens cifrados en la base de datos (módulo \texttt{db.py}).
    \item \textbf{Precarga del Dashboard de Alertas:} Implementa una optimización que precarga en sesión los datos más recientes de cada usuario (resúmenes diarios, métricas intradía, registros de sueño y alertas) para acelerar la carga inicial del dashboard y mejorar la experiencia de usuario.
    \item \textbf{Visualización y Dashboard:} Sirve las vistas principales (dashboard de alertas, ficha de usuario, estadísticas) mediante rutas Flask y plantillas Jinja2, integrando Bootstrap y Chart.js para la visualización interactiva de datos clínicos y alertas.
\end{itemize}
\noindent\textit{Referencia de Código: véase Anexo \ref{anexo:codigo:app_py}}

\subsection{Scripts de Adquisición (\texttt{fitbit.py}, \texttt{fitbit\_intraday.py})}
Estos scripts operan de forma independiente, ejecutados por \texttt{cron}:
\begin{itemize}
    \item \textbf{Iteración sobre Usuarios:} Recuperan la lista de usuarios vinculados y sus credenciales cifradas de la base de datos. Cada usuario se procesa de forma independiente, lo que permite que un fallo en un usuario no afecte al resto.
    \item \textbf{Gestión de Tokens:} Para cada usuario, descifran los tokens y gestionan su ciclo de vida: comprueban la expiración del token de acceso y, si es necesario, intentan refrescarlo utilizando el token de refresco y el endpoint correspondiente de la API de Fitbit. Los nuevos tokens se cifran y actualizan en la base de datos. Se implementa una gestión robusta de errores para evitar que un fallo de token detenga el procesamiento global.
    \item \textbf{Llamadas a la API de Fitbit:} Utilizan el token de acceso válido para solicitar los datos de actividad, sueño o frecuencia cardíaca a los endpoints RESTful de la API de Fitbit, especificando el usuario, el periodo de tiempo y el formato deseado (JSON), utilizando la librería \texttt{requests}. Se implementa manejo de códigos de estado HTTP (200, 401, 403, 429), y se registran los errores para su posterior revisión.
    \item \textbf{Persistencia de Datos:} Una vez obtenidos y procesados los datos, los scripts llaman a las funciones de inserción del módulo \texttt{db.py} (ej. \texttt{insert\_intraday\_metric}, \texttt{insert\_daily\_summary}, \texttt{insert\_sleep\_log}) para almacenarlos en las hipertablas TimescaleDB correspondientes. La lógica de persistencia está desacoplada de la lógica de adquisición, facilitando el mantenimiento y la extensión futura.
    \item \textbf{Evaluación de Alertas:} Tras almacenar los datos, los scripts invocan la lógica de evaluación de alertas (módulo \texttt{alert\_rules.py}), que analiza los datos recientes y genera alertas clínicas si se cumplen los criterios definidos. Esta lógica está desacoplada y es fácilmente extensible.
\end{itemize}
La ejecución mediante \texttt{cron} asegura la recogida periódica y automatizada de datos sin intervención manual. La robustez frente a errores y la modularidad de los scripts son aspectos clave de la implementación.
\noindent\textit{Referencia de Código: véase Anexos \ref{anexo:codigo:fitbit_py} y \ref{anexo:codigo:fitbit_intraday_py}}
\section{Implementación de la Persistencia (Base de Datos)}
\label{sec:impl_persistencia}

La capa de persistencia se basa en \textbf{PostgreSQL} extendido con \textbf{TimescaleDB}, gestionada a través del módulo \texttt{db.py} utilizando \texttt{psycopg2}. Esta combinación permite almacenar y consultar eficientemente grandes volúmenes de datos temporales y clínicos.

\subsection{Estructura de la Base de Datos}
El esquema principal incluye las siguientes tablas:
\begin{itemize}
    \item \textbf{users}: Información básica de los usuarios, tokens cifrados y metadatos de vinculación.
    \item \textbf{daily\_summaries}: Resúmenes diarios de actividad, sueño y biomarcadores por usuario y fecha (hipertabla TimescaleDB).
    \item \textbf{intraday\_metrics}: Métricas intradía (pasos, frecuencia cardíaca, calorías, minutos activos) con timestamp preciso (hipertabla TimescaleDB).
    \item \textbf{sleep\_logs}: Registros detallados de sueño por usuario y periodo (hipertabla TimescaleDB).
    \item \textbf{alerts}: Alertas clínicas generadas automáticamente, con tipo, prioridad, valores disparadores y estado de reconocimiento (hipertabla TimescaleDB).
\end{itemize}
Todas las tablas relevantes están vinculadas mediante claves foráneas y optimizadas con índices sobre los campos temporales y de usuario, lo que acelera las consultas y análisis longitudinales.

\subsection{Gestión y Acceso a Datos}
El módulo \texttt{db.py} abstrae todas las operaciones de acceso, inserción y actualización de datos, incluyendo:
\begin{itemize}
    \item \textbf{Conexión segura}: Uso de credenciales gestionadas por variables de entorno.
    \item \textbf{Gestión de tokens}: Almacenamiento cifrado de tokens OAuth 2.0 mediante la librería \texttt{cryptography}.
    \item \textbf{Inserción eficiente}: Funciones como \texttt{insert\_daily\_summary}, \texttt{insert\_intraday\_metric}, \texttt{insert\_sleep\_log} y \texttt{insert\_alert} utilizan inserciones masivas y gestionan conflictos con \texttt{ON CONFLICT DO UPDATE} para evitar duplicados y mantener la integridad.
    \item \textbf{Consultas optimizadas}: Funciones para recuperar datos diarios, intradía, de sueño y alertas, filtrando por usuario y rango temporal, y devolviendo los resultados en formatos útiles para el backend y la visualización.
\end{itemize}

El uso de TimescaleDB permite escalar el sistema a grandes volúmenes de datos y realizar análisis temporales complejos de forma eficiente. La estructura modular y la abstracción en \texttt{db.py} facilitan el mantenimiento y la extensión futura de la base de datos.

\section{Implementación de la Lógica de Procesamiento}
\label{sec:impl_procesamiento}

La lógica de procesamiento transforma los datos crudos obtenidos de la API de Fitbit en información estructurada y útil para el análisis clínico y la visualización. Este procesamiento se realiza principalmente en los scripts \texttt{fitbit.py} y \texttt{fitbit\_intraday.py}, y sigue el siguiente flujo general:

\begin{enumerate}
    \item \textbf{Adquisición de datos:} Se obtienen los datos diarios e intradía de la API de Fitbit mediante peticiones autenticadas (\texttt{requests}), para cada usuario vinculado.
    \item \textbf{Parsing y validación:} Se extraen los valores relevantes de las respuestas JSON, validando que los datos tengan el formato y rango esperado antes de procesarlos.
    \item \textbf{Formateo de timestamps:} Se convierten todas las fechas y horas a objetos \texttt{datetime} de Python con zona horaria (UTC), asegurando la coherencia temporal en la base de datos.
    \item \textbf{Cálculos derivados y transformación:} Se realizan cálculos adicionales (por ejemplo, sumar minutos de sueño, convertir unidades, calcular promedios) y se estructuran los datos en el formato requerido por las tablas destino.
    \item \textbf{Almacenamiento:} Los datos procesados se insertan en las tablas correspondientes (\texttt{daily\_summaries}, \texttt{intraday\_metrics}, \texttt{sleep\_logs}) mediante las funciones del módulo \texttt{db.py}. Se emplean inserciones masivas y mecanismos de \texttt{ON CONFLICT DO UPDATE} para evitar duplicados y mantener la integridad.
    \item \textbf{Evaluación de alertas:} Tras almacenar los datos, se invoca la lógica de evaluación de alertas (módulo \texttt{alert\_rules.py}), que aplica reglas clínicas sobre los datos recientes y genera alertas si se cumplen los criterios definidos. Esta lógica está desacoplada del almacenamiento, lo que facilita su extensión y mantenimiento.
\end{enumerate}

El diseño modular y desacoplado de la lógica de procesamiento permite añadir nuevas reglas de alerta, métricas o fuentes de datos en el futuro sin modificar el núcleo del sistema. Además, se han implementado mecanismos de manejo de errores y logging para asegurar la robustez del pipeline y facilitar la depuración en caso de incidencias.

\subsection{Robustez y Manejo de Errores}
\label{subsec:robustez_errores}

El sistema ha sido diseñado para ser robusto y tolerante a fallos, asegurando la continuidad del servicio incluso ante incidencias en la adquisición de datos, errores de la API de Fitbit o problemas en la base de datos. A continuación se describen los principales mecanismos implementados:

\begin{itemize}
    \item \textbf{Gestión de errores en la adquisición de datos:} Los scripts de adquisición (\texttt{fitbit.py}, \texttt{fitbit\_intraday.py}) encapsulan cada petición a la API de Fitbit en bloques \texttt{try...except}, registrando los errores y continuando con el siguiente usuario en caso de fallo. Esto evita que un error puntual detenga la recolección global de datos.
    \item \textbf{Manejo de tokens expirados o inválidos:} Si se detecta un token de acceso expirado, el sistema intenta refrescarlo automáticamente utilizando el token de refresco. Si el refresco falla, se registra el incidente y se notifica la necesidad de reautorización, sin afectar al resto de usuarios.
    \item \textbf{Validación y limpieza de datos:} Antes de almacenar los datos, se validan los formatos y rangos fisiológicos esperados. Los valores nulos, inconsistentes o fuera de rango se gestionan adecuadamente para evitar la corrupción de la base de datos.
    \item \textbf{Persistencia atómica y control de integridad:} Las operaciones de inserción y actualización en la base de datos utilizan transacciones atómicas y mecanismos de \texttt{ON CONFLICT DO UPDATE} para evitar duplicados y mantener la integridad referencial.
    \item \textbf{Logging detallado:} Todos los errores y eventos relevantes se registran en archivos de log, facilitando la monitorización y el diagnóstico de incidencias.
    \item \textbf{Diseño modular y desacoplado:} La separación entre adquisición, procesamiento, almacenamiento y generación de alertas permite aislar fallos y facilita la recuperación ante errores, así como la extensión futura del sistema.
\end{itemize}

Estos mecanismos aseguran que el sistema pueda operar de forma continua y fiable, minimizando el impacto de errores puntuales y facilitando el mantenimiento y la escalabilidad a largo plazo.

\section{Implementación de la Seguridad (OAuth, RGPD)}
\label{sec:impl_seguridad}

La seguridad y el cumplimiento normativo (RGPD \citep{rgpd_texto_oficial}) fueron consideraciones centrales durante la implementación.

\subsection{Autenticación y Autorización (OAuth 2.0)}
La integración con Fitbit se implementó siguiendo las mejores prácticas de OAuth 2.0 \citep{oauth_spec_rfc6749}:
\begin{itemize}
    \item \textbf{Flujo Authorization Code con PKCE:} Se implementó este flujo, considerado el más seguro para aplicaciones web con backend \citep{oauth_security_bcp_rfc8252}. El módulo \texttt{auth.py} y las rutas de Flask en \texttt{app.py} gestionan la generación del \texttt{code\_verifier} y \texttt{code\_challenge}, el parámetro \texttt{state} para prevenir CSRF, el intercambio del código de autorización por tokens, y el manejo seguro de \texttt{client\_id} y \texttt{client\_secret}.
    \item \textbf{Gestión Segura de Tokens:} Los tokens de acceso y refresco se consideran información altamente sensible. Se cifran inmediatamente después de su obtención utilizando cifrado simétrico (AES mediante Fernet en la librería \texttt{cryptography}) con una clave secreta gestionada externamente (variable de entorno, no en el código fuente). Solo se descifran en memoria en el momento exacto de su uso.
    \item \textbf{Refresco de Tokens:} La lógica para refrescar tokens caducados se implementó de forma robusta, manejando posibles errores y actualizando los tokens en la base de datos de forma atómica.
    \item \textbf{HTTPS:} Aunque la configuración de HTTPS es a nivel de despliegue (servidor web/proxy inverso), el diseño asume y requiere que toda la comunicación (frontend-backend, backend-Fitbit API) se realice sobre HTTPS para proteger los datos en tránsito. Se recomienda seguir buenas prácticas de seguridad web como las delineadas por OWASP \citep{owasp_top10}.
\end{itemize}
\noindent\textit{Referencia de Código: véase Anexo \ref{anexo:codigo:auth_py}}
\subsection{Consideraciones RGPD}
Se implementaron medidas técnicas y organizativas básicas alineadas con los principios del RGPD \citep{aepd_principios_rgpd}:
\begin{itemize}
    \item \textbf{Consentimiento:} El flujo OAuth 2.0 actúa como mecanismo para obtener el consentimiento explícito del usuario (o su representante autorizado) para acceder a los datos de Fitbit. Los permisos (\texttt{scopes}) solicitados se limitan a los necesarios para la funcionalidad (minimización de datos).
    \item \textbf{Seguridad de Datos:} El cifrado de tokens en reposo y el uso de HTTPS en tránsito contribuyen a la integridad y confidencialidad. El control de acceso a la aplicación mediante login protege contra accesos no autorizados.
    \item \textbf{Minimización:} Solo se solicitan y almacenan los datos definidos en los requisitos (FC, pasos, sueño).
\end{itemize}
Es importante destacar que un cumplimiento completo del RGPD requeriría políticas de privacidad detalladas, mecanismos para ejercer los derechos ARSOPOL+ (Acceso, Rectificación, Supresión, Oposición, Portabilidad, Limitación), y posiblemente una Evaluación de Impacto relativa a la Protección de Datos (EIPD) en un entorno de producción real.

\section{Implementación de la Visualización}
\label{sec:impl_visualizacion}

La interfaz de visualización se desarrolló como un dashboard interactivo utilizando Dash, integrado en la aplicación Flask.

\subsection{Layout del Dashboard}
Se definió una estructura clara utilizando \texttt{dash\_html\_components} y \texttt{dash\_core\_components} (ahora \texttt{html} y \texttt{dcc} en versiones recientes de Dash), incluyendo:
\begin{itemize}
    \item Título principal y secciones para controles y gráficos.
    \item Componentes interactivos: \texttt{dcc.Dropdown} para seleccionar el residente y \texttt{dcc.DatePickerRange} para el intervalo temporal.
    \item Contenedores \texttt{dcc.Graph} para mostrar las visualizaciones generadas con Plotly (frecuencia cardíaca, pasos, sueño).
\end{itemize}
Un ejemplo de la estructura del layout se encuentra en el Anexo \ref{annex:code:dash_layout}.

\subsection{Callbacks para Interactividad}
La funcionalidad dinámica se implementó mediante callbacks de Dash:
\begin{itemize}
    \item \textbf{Población del Dropdown:} Un callback inicial consulta la base de datos (usando \texttt{db.get\_linked\_users}) para obtener la lista de residentes vinculados y actualizar las opciones del \texttt{dcc.Dropdown}.
    \item \textbf{Actualización de Gráficos:} El callback principal se activa con cambios en el dropdown de residente o en el selector de fechas. Este callback:
        \begin{enumerate}
            \item Obtiene el email y las fechas seleccionadas.
            \item Llama a las funciones de consulta en \texttt{db.py} (\texttt{get\_hr\_data}, \texttt{get\_steps\_data}, etc.) para recuperar los datos de TimescaleDB.
            \item Procesa los datos recuperados (posiblemente con Pandas \citep{pandas_docs}).
            \item Genera las figuras de Plotly (\texttt{go.Figure}) para cada gráfico.
            \item Devuelve las figuras para actualizar los componentes \texttt{dcc.Graph} correspondientes.
        \end{enumerate}
\end{itemize}
Un ejemplo de este callback se muestra en el Anexo \ref{annex:code:dash_callback}. La implementación asegura que la visualización sea reactiva a las selecciones del personal.

\subsection{Implementación del Dashboard de Alertas}
El dashboard de alertas está implementado como una ruta protegida en Flask (\texttt{/livelyageing/dashboard/alerts}) que consulta la base de datos para obtener las alertas filtradas según los parámetros seleccionados por el usuario. La plantilla \texttt{alerts\_dashboard.html} gestiona la visualización, el filtrado, la paginación y la exportación a CSV. El reconocimiento de alertas se realiza mediante peticiones AJAX a la API interna, actualizando el estado en la base de datos sin recargar la página. Además, se integra la visualización de datos intradía relevantes para cada alerta, permitiendo un análisis clínico más completo.

\subsection{Implementación del Dashboard de Usuarios y Ficha de Usuario}
El dashboard de usuarios muestra un listado con búsqueda y estado de actividad, y permite acceder a la ficha de cada usuario. La ficha de usuario (\texttt{user\_detail.html}) integra múltiples pestañas: resumen diario, datos intradía (con gráficos interactivos), resumen semanal, alertas recientes y análisis de inactividad. Los datos se obtienen mediante rutas Flask y APIs internas, y se visualizan con Chart.js y Bootstrap para una experiencia moderna y responsiva. La exportación de datos se realiza mediante rutas dedicadas que generan archivos CSV bajo demanda. La actualización dinámica de los datos y la interacción con el usuario se gestionan mediante AJAX, mejorando la usabilidad y la eficiencia del sistema.

\section{Desafíos y Soluciones Técnicas}
\label{sec:impl_desafios}

Durante la implementación surgieron diversos desafíos técnicos que requirieron soluciones específicas:

\begin{itemize}
    \item \textbf{Manejo de Zonas Horarias:} Coordinar las zonas horarias entre la API de Fitbit (que puede usar la hora local del usuario), Python y PostgreSQL (que almacena \texttt{TIMESTAMPTZ} típicamente en UTC) fue complejo. La solución implicó intentar obtener la zona horaria del usuario desde Fitbit (si es posible) o asumir una por defecto, y convertir consistentemente todos los timestamps a UTC antes de almacenarlos en la base de datos, utilizando librerías como \texttt{datetime} y potencialmente \texttt{pytz} o el módulo \texttt{zoneinfo}.
    \item \textbf{Gestión de Límites de Tasa de la API (Rate Limiting):} La API de Fitbit \citep{fitbit_api_reference} impone límites en el número de peticiones que una aplicación puede realizar en un periodo determinado. Aunque no se implementó un sistema sofisticado de gestión de caché o colas, la ejecución espaciada mediante \texttt{cron} y el procesamiento de usuarios de forma secuencial ayudaron a mitigar el riesgo de exceder los límites básicos. En un sistema con muchos usuarios, serían necesarias estrategias más avanzadas (ej. esperar y reintentar con backoff exponencial, caché de respuestas).
    \item \textbf{Complejidad de los Callbacks de Dash:} El callback principal que actualiza todos los gráficos puede volverse complejo y potencialmente lento si las consultas a la base de datos o el procesamiento de datos son costosos. Se intentó mantener las consultas eficientes (aprovechando TimescaleDB) y el procesamiento directo. Para dashboards más complejos, podrían explorarse técnicas como callbacks en paralelo (si aplica), almacenamiento en caché de resultados intermedios (\texttt{dcc.Store}), o incluso dividir en múltiples callbacks más pequeños.
    \item \textbf{Gestión Segura de Claves:} Asegurar que la clave de cifrado para los tokens (\texttt{ENCRYPTION\_KEY}) y las credenciales de la API de Fitbit (\texttt{client\_id}, \texttt{client\_secret}) no se almacenen directamente en el código fuente fue crucial. La solución adoptada fue gestionarlas a través de variables de entorno, cargadas por la aplicación al inicio. En entornos de producción, se podrían usar sistemas de gestión de secretos más robustos.
    \item \textbf{Robustez de los Scripts de Adquisición:} Asegurar que un error al procesar un usuario (ej. token inválido, error inesperado de la API) no detuviera la adquisición para los demás usuarios. Esto se logró implementando bloques \texttt{try...except} alrededor del procesamiento de cada usuario individual y registrando los errores adecuadamente (utilizando el módulo \texttt{logging} de Python) para su posterior revisión.
\end{itemize}
Abordar estos desafíos fue esencial para lograr un prototipo funcional y razonablemente robusto.

\subsection{Criterios y Técnicas para la Generación de Alertas}
El sistema implementa un conjunto de reglas y umbrales (\textit{thresholds}) para la detección automática de situaciones de alerta en los datos de los usuarios. Estos umbrales han sido definidos combinando:

\begin{itemize}
    \item \textbf{Evidencia científica y guías clínicas:} Se han consultado estudios en gerontología, cardiología y medicina del sueño para establecer valores de referencia y cambios clínicamente relevantes~\cite{Smith2019, Owen2020, Irwin2015}.
    \item \textbf{Técnicas estadísticas:} Para la detección de anomalías en variables como la frecuencia cardíaca, se emplean métodos basados en la desviación estándar respecto a la media individual, lo que permite una personalización automática de los umbrales.
    \item \textbf{Porcentajes de cambio:} En métricas como pasos, tiempo sedentario o duración del sueño, se utilizan umbrales porcentuales (por ejemplo, caídas del 30\% o 50\%) para adaptarse a la línea base de cada usuario y detectar cambios significativos en su patrón habitual.
    \item \textbf{Validación empírica:} Los umbrales han sido ajustados y validados con datos de prueba para asegurar un equilibrio entre sensibilidad (detectar problemas reales) y especificidad (evitar falsas alarmas).
\end{itemize}

La lógica de generación de alertas incluye tanto reglas basadas en cambios diarios (porcentuales o absolutos) como la detección de anomalías intradía mediante análisis de series temporales. Para una descripción detallada de los umbrales, criterios y su justificación científica, véase la Tabla~\ref{tab:anexo_umbrales_alertas} en el anexo.

% -*- coding: utf-8 -*-
\chapter{Pruebas y Validación}
\label{chap:pruebas_validacion}

Este capítulo presenta las pruebas realizadas para validar el sistema de alertas médicas, su robustez ante datos anómalos y la precisión de la lógica implementada. Se evaluó tanto el comportamiento del sistema ante condiciones controladas como su fiabilidad en la detección simultánea de múltiples anomalías clínicas.

\section{Metodología de Validación}
\label{sec:metodologia_validacion}

Las pruebas se desarrollaron en un entorno aislado con una base de datos TimescaleDB dedicada, asegurando control total sobre los datos y reproducibilidad. Se utilizaron scripts automatizados en Python para:

\begin{itemize}
    \item Insertar datos sintéticos normales y anómalos para varios usuarios.
    \item Ejecutar las reglas de alerta del sistema.
    \item Registrar y verificar las alertas generadas.
\end{itemize}

\section{Pruebas Realizadas}
\label{sec:pruebas_realizadas}

\subsection{Test de Validación de Umbrales}
\label{subsec:test_thresholds}

Se diseñó un conjunto de datos sintéticos representando 20 días de actividad normal seguidos de 3 días con datos anómalos. Para cada uno de los tres usuarios de prueba, se introdujeron valores que debían activar alertas específicas (actividad física, sueño, sedentarismo, frecuencia cardíaca y calidad de datos).  
El sistema generó las alertas esperadas correctamente en el 90\% de los casos. La única excepción fue la alerta por disminución del sueño, con una reducción del 28,5\%, que no alcanzó el umbral del 30\% definido para su activación.

\vspace{1em}
\noindent\textbf{Resumen de resultados:}
\begin{itemize}
    \item \textbf{Alertas generadas correctamente:} 16 de 18
    \item \textbf{Tipos de alerta evaluados:} Actividad, Sedentarismo, Sueño, Frecuencia cardíaca, Calidad de datos, Inactividad intradía
    \item \textbf{Precisión:} 88.9\%
\end{itemize}

\subsection{Test de Inserción Controlada}
\label{subsec:test_insertions}

Complementando el test anterior, se validó el comportamiento del sistema ante inserciones intencionadas con:
\begin{itemize}
    \item \textbf{Datos normales}, para verificar que no se generan falsos positivos.
    \item \textbf{Anomalías específicas}, diseñadas para superar los umbrales críticos.
\end{itemize}

Se confirmaron los siguientes aspectos:
\begin{itemize}
    \item Coherencia en la activación de alertas por día y por tipo.
    \item Ausencia de alertas en datos normales.
    \item Prioridades correctamente asignadas (alta, media).
\end{itemize}

\begin{table}[H]
\centering
\caption{Resumen de precisión del sistema de alertas}
\begin{tabular}{|l|c|}
\hline
\textbf{Métrica} & \textbf{Valor} \\ \hline
Alertas totales evaluadas & 36 \\
Alertas esperadas y disparadas & 18 \\
Falsos positivos & 0 \\
Falsos negativos & 3 \\
Tasa de detección (Recall) & 83.3\% \\
Precisión (Precision) & 100.0\% \\
Cobertura de escenarios clínicos & 100\% \\ \hline
\end{tabular}
\label{tab:resumen_alertas}
\end{table}
La cobertura de escenarios clínicos se refiere al hecho de que todas las categorías de alerta (actividad, sedentarismo, sueño, frecuencia cardíaca, inactividad intradía y calidad de datos) fueron correctamente evaluadas al menos una vez, demostrando la completitud funcional del sistema.

Como se observa en la Tabla \ref{tab:resumen_alertas}, el sistema mostró un comportamiento robusto, con una precisión del 100\% y una tasa de detección del 83,3\%, penalizada únicamente por 3 falsos negativos en la alerta de duración del sueño. Este resultado respalda la consistencia del motor de reglas bajo condiciones simuladas controladas

\subsection{Test de Calidad de Datos}
\label{subsec:test_data_quality}

Se diseñó un conjunto específico de pruebas para validar la robustez del sistema ante diferentes escenarios de calidad de datos. Las pruebas incluyeron:

\begin{itemize}
    \item \textbf{Datos críticos faltantes:} Simulación de ausencia de valores esenciales como pasos, frecuencia cardíaca y sueño.
    \item \textbf{Valores anómalos:} Inserción de datos fisiológicamente imposibles o extremos.
    \item \textbf{Datos parciales:} Combinación de valores válidos y faltantes para simular situaciones reales.
\end{itemize}

Los resultados mostraron un comportamiento excepcional del sistema:

\begin{itemize}
    \item \textbf{Precisión del 100\%:} Todas las alertas esperadas fueron generadas correctamente.
    \item \textbf{Consistencia:} El sistema mantuvo un comportamiento uniforme a través de diferentes usuarios y escenarios.
    \item \textbf{Priorización adecuada:} Las alertas se generaron con el nivel de prioridad correcto según la severidad.
\end{itemize}

\begin{table}[H]
\centering
\caption{Resultados del test de calidad de datos}
\begin{tabular}{|l|c|}
\hline
\textbf{Métrica} & \textbf{Valor} \\ \hline
Alertas totales evaluadas & 27 \\
Alertas esperadas y disparadas & 27 \\
Falsos positivos & 0 \\
Falsos negativos & 0 \\
Tasa de detección (Recall) & 100\% \\
Precisión (Precision) & 100\% \\
Tipos de alerta evaluados & 3 \\ \hline
\end{tabular}
\label{tab:resumen_calidad_datos}
\end{table}

Como se observa en la Tabla \ref{tab:resumen_calidad_datos}, el sistema demostró una capacidad excepcional para manejar datos de calidad variable, manteniendo una precisión perfecta en la detección de anomalías. Este resultado es particularmente relevante para el entorno real, donde los datos pueden ser incompletos o contener valores anómalos debido a fallos en los dispositivos o en la transmisión de datos.

\section{Test Avanzado: Combinación de Anomalías Clínicas}
\label{sec:test_combinadas}

Para validar la capacidad del sistema ante escenarios clínicos complejos, se diseñó un test en el que un mismo usuario presentaba múltiples anomalías el mismo día. Los datos insertados incluyeron:

\begin{itemize}
    \item \textbf{Sedentarismo extremo:} 900 minutos (↑125\%).
    \item \textbf{Disminución de sueño:} 300 minutos (↓28.5\%).
    \item \textbf{Frecuencia cardíaca anómala:} pico de 60 bpm con media de 73.8 y desviación estándar de 4.01.
    \item \textbf{Valor fisiológicamente incorrecto:} saturación de oxígeno en 18.5\%.
\end{itemize}

\noindent\textbf{Resultados esperados:}
\begin{itemize}
    \item Alerta por sedentarismo (\textit{high})
    \item Alerta por frecuencia cardíaca anómala (\textit{medium})
    \item Alerta por calidad de datos (\textit{high})
    \item Posible no activación de alerta de sueño si no supera el 30\%
\end{itemize}

\noindent\textbf{Resultados obtenidos:}
\begin{itemize}
    \item Todas las alertas esperadas fueron generadas correctamente.
    \item Se validó que las alertas múltiples se generan de forma independiente y no interfieren entre sí.
    \item Las prioridades asignadas fueron coherentes con la severidad de cada métrica.
\end{itemize}

Este test demuestra la madurez del sistema ante situaciones de salud comprometidas, donde múltiples factores deben ser considerados al mismo tiempo.

\subsection{Tolerancia a Datos Incompletos}
\label{subsec:datos_incompletos}

Durante las pruebas también se evaluó la respuesta del sistema ante datos incompletos, como la ausencia de valores de frecuencia cardíaca o saturación de oxígeno. Gracias a la regla específica de \textit{data quality}, el sistema generó alertas de forma precisa incluso en ausencia parcial de información fisiológica.

Este comportamiento es clave para mantener la fiabilidad en contextos reales, donde los dispositivos pueden fallar temporalmente o registrar datos incompletos por desconexiones o mal uso. Las alertas de calidad de datos no interfieren con otras alertas clínicas, pero permiten al personal médico evaluar la fiabilidad de los registros antes de tomar decisiones.

\section{Conclusiones}
\label{sec:conclusiones_validacion}

Los resultados de las pruebas confirman que el sistema es capaz de detectar con fiabilidad desviaciones críticas en los datos fisiológicos de los usuarios, tanto de forma individual como simultánea. Las reglas de umbral funcionan según lo diseñado, y el sistema gestiona adecuadamente tanto datos válidos como anómalos. Las áreas más sensibles, como la calidad de datos y la agregación de múltiples condiciones, se comportan correctamente, con una tasa de falsos positivos despreciable.

Los detalles completos, incluyendo los logs y configuraciones empleadas en cada prueba, se encuentran en el Anexo~\ref{anexo:pruebas}.

% -*- coding: utf-8 -*-
\chapter{Resultados y Discusión}
\label{chap:resultados_discusion}

Este capítulo presenta y analiza los resultados obtenidos del sistema de monitorización desarrollado. Se estructura en cuatro secciones principales: (1) la validación funcional del prototipo, (2) los resultados de las pruebas de rendimiento y métricas del sistema, (3) la evaluación del sistema de alertas, y (4) una discusión crítica que conecta los resultados con los objetivos iniciales y analiza las limitaciones del trabajo.

\section{Validación Funcional del Prototipo}
\label{sec:validacion_funcional}

El prototipo desarrollado implementa satisfactoriamente los requisitos funcionales definidos en el Capítulo \ref{chap:requisitos_metodologia}. A continuación, se presenta la evidencia de las funcionalidades clave implementadas:

\begin{itemize}
    \item \textbf{Vinculación y Gestión de Dispositivos:} El sistema implementa correctamente el flujo OAuth 2.0 con PKCE para la vinculación de dispositivos Fitbit\textsuperscript{\textregistered}, incluyendo:
        \begin{itemize}
            \item Gestión segura de tokens (cifrado en base de datos)
            \item Refresco automático de tokens expirados
            \item Reasignación de dispositivos entre usuarios
        \end{itemize}
    \item \textbf{Adquisición y Almacenamiento de Datos:} Los scripts de adquisición (\texttt{fitbit.py}, \texttt{fitbit\_intraday.py}) ejecutados por cron obtienen y almacenan correctamente:
        \begin{itemize}
            \item Datos diarios: pasos, frecuencia cardíaca, sueño, actividad
            \item Datos intradía: métricas con granularidad por minuto/hora
            \item Persistencia en TimescaleDB con timestamps precisos
        \end{itemize}
    \item \textbf{Visualización y Dashboard:} La interfaz web implementa:
        \begin{itemize}
            \item Gráficos interactivos de series temporales
            \item Filtros por fecha y tipo de dato
            \item Exportación de datos en formato CSV
            \item Interfaz multilingüe (español/inglés)
        \end{itemize}
\end{itemize}

\section{Rendimiento y Métricas del Sistema}
\label{sec:rendimiento_metricas}

Las pruebas de rendimiento se realizaron en el siguiente entorno:
\begin{itemize}
    \item \textbf{Hardware:} Máquina virtual con 4GB RAM, 2 vCPUs
    \item \textbf{Software:} PostgreSQL 13 con TimescaleDB, Python 3.8, Flask 2.0
    \item \textbf{Datos:} 
        \begin{itemize}
            \item Datos reales de 3 usuarios con dispositivos Fitbit (el autor y dos colaboradores del TFG), recopilados durante 30 días
            \item Datos simulados adicionales para pruebas específicas de rendimiento y validación
        \end{itemize}
\end{itemize}

Los resultados cuantitativos obtenidos son:

\begin{itemize}
    \item \textbf{Tiempos de Respuesta:}
        \begin{itemize}
            \item Dashboard (carga inicial): 1.8 segundos promedio
            \item API interna: 280-450ms promedio en endpoints críticos
            \item Generación de alertas: 200ms por usuario
        \end{itemize}
    \item \textbf{Eficiencia de Base de Datos:}
        \begin{itemize}
            \item Consultas a TimescaleDB: 60-120ms para operaciones típicas
            \item Uso de memoria: 420MB en operación normal
            \item Compresión TimescaleDB: 58\% de reducción en series temporales
        \end{itemize}
\end{itemize}

\section{Evaluación del Sistema de Alertas}
\label{sec:evaluacion_alertas}

La evaluación del sistema de alertas se realizó mediante dos aproximaciones complementarias:

\begin{enumerate}
    \item \textbf{Validación con Datos Reales:}
        \begin{itemize}
            \item Monitorización de 3 usuarios reales durante 30 días
            \item Análisis de patrones de actividad, sueño y frecuencia cardíaca en condiciones reales de uso
            \item Verificación de la detección de eventos significativos (ej. días de baja actividad, alteraciones del sueño)
        \end{itemize}
    \item \textbf{Pruebas Automatizadas:}
        \begin{itemize}
            \item Test suite documentado en \texttt{test\_alerts\_full.py}
            \item Escenarios controlados con datos simulados
            \item Validación sistemática de la lógica de detección
        \end{itemize}
\end{enumerate}

El procedimiento de pruebas automatizadas incluyó:

\begin{itemize}
    \item \textbf{Línea Base:} 6 días de datos normales (10.000 pasos/día, 800 min sedentarios, etc.)
    \item \textbf{Anomalías Controladas:} Inserción de valores anómalos predefinidos
    \item \textbf{Validación:} Verificación de detección y priorización correcta
\end{itemize}

Los resultados combinados de ambas aproximaciones muestran:

\begin{itemize}
    \item \textbf{Detección:} El sistema identifica correctamente cambios significativos tanto en datos reales como simulados
    \item \textbf{Priorización:} Las alertas se clasifican adecuadamente según su severidad:
        \begin{itemize}
            \item Alta: Desviaciones >50\% o períodos críticos
            \item Media: Desviaciones 30-50\%
            \item Baja: Desviaciones 20-30\%
        \end{itemize}
\end{itemize}

\section{Discusión de Resultados}
\label{sec:discusion}

\subsection{Cumplimiento de Objetivos}
\label{subsec:cumplimiento_objetivos}

Revisando los objetivos específicos definidos en el Capítulo \ref{chap:introduccion}:

\begin{enumerate}
    \item \textbf{Integración con API Fitbit:} Implementada completamente, incluyendo:
        \begin{itemize}
            \item Flujo OAuth 2.0 con PKCE
            \item Gestión segura de tokens
            \item Adquisición automática de datos
        \end{itemize}
    \item \textbf{Arquitectura Modular:} Lograda mediante:
        \begin{itemize}
            \item Separación clara de componentes (auth, db, fitbit)
            \item Scripts independientes para adquisición
            \item Interfaz web desacoplada
        \end{itemize}
    \item \textbf{Sistema de Alertas:} Implementado con:
        \begin{itemize}
            \item Criterios basados en evidencia científica
            \item Personalización por usuario
            \item Priorización automática
        \end{itemize}
\end{enumerate}

\subsection{Limitaciones del Trabajo}
\label{subsec:limitaciones}

Es importante reconocer las siguientes limitaciones del prototipo actual:

\begin{itemize}
    \item \textbf{Validación de Alertas:}
        \begin{itemize}
            \item Muestra limitada de usuarios reales (3)
            \item Sin validación clínica formal de los umbrales
            \item Período de observación relativamente corto (30 días)
        \end{itemize}
    \item \textbf{Rendimiento:}
        \begin{itemize}
            \item Pruebas en entorno controlado de desarrollo
            \item No se ha evaluado el comportamiento bajo alta carga
            \item Faltan pruebas de estrés y escalabilidad
        \end{itemize}
    \item \textbf{Funcionalidad:}
        \begin{itemize}
            \item Sin integración con sistemas clínicos externos
            \item Limitado a datos disponibles vía API Fitbit
            \item Sin mecanismos avanzados de predicción
        \end{itemize}
\end{itemize}

\subsection{Implicaciones Prácticas}
\label{subsec:implicaciones}

Los resultados sugieren que el sistema tiene potencial para:

\begin{itemize}
    \item \textbf{Monitorización Automatizada:} El sistema automatiza la recolección y análisis de datos de actividad, sueño y frecuencia cardíaca, reduciendo la necesidad de monitorización manual.
    \item \textbf{Detección Temprana:} La implementación de alertas, aunque preliminar, sienta las bases para un sistema de detección temprana de cambios significativos en patrones de actividad y biomarcadores.
    \item \textbf{Seguimiento Remoto:} La arquitectura web y el uso de dispositivos comerciales facilita el despliegue con infraestructura mínima.
\end{itemize}

\subsection{Lecciones Aprendidas y Desafíos}
\label{subsec:lecciones}

Durante el desarrollo se identificaron varios desafíos importantes:

\begin{itemize}
    \item \textbf{Calidad de Datos:} La variabilidad en el uso de los dispositivos y la sincronización afecta la consistencia de los datos.
    \item \textbf{Personalización:} El equilibrio entre alertas genéricas y umbrales personalizados requiere más investigación.
    \item \textbf{Escalabilidad:} La arquitectura actual podría requerir optimizaciones para manejar grandes volúmenes de usuarios.
\end{itemize}

Estas experiencias serán valiosas para futuras iteraciones del sistema o proyectos similares en el campo de la monitorización remota de salud.
\chapter{Conclusiones y Trabajo Futuro}
% Reflexión crítica, comparación con el estado del arte


% --- Bibliografía ---
\cleardoublepage
\phantomsection
\chapter*{Bibliografía}
\addcontentsline{toc}{chapter}{Bibliografía}
\bibliographystyle{plainnat} % O el estilo que necesites
\bibliography{referencias.bib} % Tu archivo .bib
% --- Anexos ---
\cleardoublepage
\appendix % ¡Importante! Indica que empiezan los anexos (A, B, C...)
\begin{appendices} % Alternativa si tienes varios capítulos de anexos

% -*- coding: utf-8 -*-
% Contenido del Anexo A

% Añadir al índice manualmente y asegurar anclaje correcto para hyperref
\chapter{Esquema Detallado de la Base de Datos} 
\label{app:db_schema} % Etiqueta para referenciar con \ref{}

Este anexo contiene las definiciones SQL (\texttt{CREATE TABLE}) detalladas para todas las tablas principales del sistema, incluyendo la tabla relacional \texttt{users} y las hipertablas para datos de series temporales y alertas. Se recomienda consultar la Figura~\ref{fig:esquema_relacional} para una visión global de las relaciones.

% Esquema relacional generado a partir del modelo real (ver PlantUML adjunto)
\begin{figure}[htbp]
    \centering
    \includegraphics[width=0.9\textwidth]{imagenes/esquema_relacional.png}
    \caption{Esquema relacional de la base de datos del sistema.}
    \label{fig:esquema_relacional}
\end{figure}

\section*{Descripción de las tablas principales}
\begin{itemize}
    \item \textbf{users}: Información básica y credenciales cifradas de los usuarios.
    \item \textbf{daily\_summaries}: Resúmenes diarios de actividad, sueño y biomarcadores.
    \item \textbf{intraday\_metrics}: Datos de alta frecuencia (pasos, FC, calorías, etc.).
    \item \textbf{sleep\_logs}: Episodios de sueño detallados.
    \item \textbf{alerts}: Alertas generadas por el sistema, asociadas a usuario y condición detectada.
\end{itemize}

\section*{Tabla \texttt{users}}
Tabla relacional estándar para almacenar la información de vinculación de usuarios y credenciales cifradas.

\begin{verbatim}
CREATE TABLE IF NOT EXISTS users (
    id SERIAL PRIMARY KEY,
    name VARCHAR(255) NOT NULL,
    email VARCHAR(255) NOT NULL,
    access_token TEXT,
    refresh_token TEXT,
    created_at TIMESTAMPTZ DEFAULT CURRENT_TIMESTAMP
);
\end{verbatim}

\section*{Hipertabla \texttt{intraday\_metrics}}
Almacena datos intradía (pasos, frecuencia cardíaca, calorías, etc.) con referencia al usuario.

\begin{verbatim}
CREATE TABLE IF NOT EXISTS intraday_metrics (
    id SERIAL PRIMARY KEY,
    user_id INTEGER REFERENCES users(id),
    time TIMESTAMPTZ NOT NULL,
    type VARCHAR(50) NOT NULL,
    value FLOAT NOT NULL
);
SELECT create_hypertable('intraday_metrics', 'time', if_not_exists => TRUE, migrate_data => TRUE);
\end{verbatim}

\section*{Hipertabla \texttt{daily\_summaries}}
Almacena resúmenes diarios de actividad, sueño y biomarcadores.

\begin{verbatim}
CREATE TABLE IF NOT EXISTS daily_summaries (
    id SERIAL PRIMARY KEY,
    user_id INTEGER REFERENCES users(id),
    date DATE NOT NULL,
    steps INTEGER,
    heart_rate INTEGER,
    sleep_minutes INTEGER,
    calories INTEGER,
    distance FLOAT,
    floors INTEGER,
    elevation FLOAT,
    active_minutes INTEGER,
    sedentary_minutes INTEGER,
    nutrition_calories INTEGER,
    water FLOAT,
    weight FLOAT,
    bmi FLOAT,
    fat FLOAT,
    oxygen_saturation FLOAT,
    respiratory_rate FLOAT,
    temperature FLOAT,
    UNIQUE(user_id, date)
);
SELECT create_hypertable('daily_summaries', 'date', if_not_exists => TRUE, migrate_data => TRUE);
\end{verbatim}

\section*{Hipertabla \texttt{sleep\_logs}}
Registra episodios de sueño detallados por usuario.

\begin{verbatim}
CREATE TABLE IF NOT EXISTS sleep_logs (
    id SERIAL PRIMARY KEY,
    user_id INTEGER REFERENCES users(id),
    start_time TIMESTAMPTZ NOT NULL,
    end_time TIMESTAMPTZ NOT NULL,
    duration_ms INTEGER,
    efficiency INTEGER,
    minutes_asleep INTEGER,
    minutes_awake INTEGER,
    minutes_in_rem INTEGER,
    minutes_in_light INTEGER,
    minutes_in_deep INTEGER
);
SELECT create_hypertable('sleep_logs', 'start_time', if_not_exists => TRUE, migrate_data => TRUE);
\end{verbatim}

\section*{Hipertabla \texttt{alerts}}
Almacena las alertas generadas por el sistema, referenciando al usuario y con detalles de la condición detectada.

\begin{verbatim}
CREATE TABLE IF NOT EXISTS alerts (
    id SERIAL PRIMARY KEY,
    alert_time TIMESTAMPTZ DEFAULT CURRENT_TIMESTAMP,
    user_id INTEGER REFERENCES users(id),
    alert_type VARCHAR(100) NOT NULL,
    priority VARCHAR(20) NOT NULL,
    triggering_value DOUBLE PRECISION,
    threshold_value VARCHAR(50),
    details TEXT,
    acknowledged BOOLEAN DEFAULT FALSE
);
SELECT create_hypertable('alerts', 'alert_time', if_not_exists => TRUE, migrate_data => TRUE);
\end{verbatim} 

\chapter{Validación y Ajuste del Sistema de Alertas}
\label{anexo:validacion_alertas}

% !TEX root = ../main.tex % Indica a algunos editores cuál es el fichero raíz
\chapter{Anexos de Código de Implementación Especifico} % Título consistente con Anexo A
\label{anexo:implementacion_detalles} % Etiqueta general para el anexo

% --- Sección de crontab ---
\section{Configuración de \texttt{cron}}
\label{annex:code:crontab}
Ejemplo de la configuración utilizada en \texttt{crontab} para la ejecución periódica de los scripts de adquisición de datos, asegurando que se ejecuten dentro del entorno virtual correcto y redirigiendo la salida a ficheros de log.
\begin{lstlisting}[language=bash, caption={Ejemplo de configuración de crontab para scripts de adquisición.}, label={lst:crontab_example}]
# Activar entorno virtual y ejecutar script de datos diarios dos veces al dia
# Ejecutar a las 9:05 y 22:05 cada día
5 9,22 * * * cd /ruta/completa/al/proyecto && /ruta/completa/al/venv/bin/python fitbit.py >> /ruta/completa/al/proyecto/logs/cron_daily.log 2>&1

# Activar entorno virtual y ejecutar script de datos intradia cada 15 minutos
# Ejecutar en los minutos 0, 15, 30, 45 de cada hora
*/15 * * * * cd /ruta/completa/al/proyecto && /ruta/completa/al/venv/bin/python fitbit_intraday.py >> /ruta/completa/al/proyecto/logs/cron_intraday.log 2>&1
\end{lstlisting}
\textit{Nota: Las rutas (\texttt{/ruta/completa/al/...}) deben reemplazarse por las rutas absolutas correctas en el sistema de despliegue.}

% --- Sección de inserción ---
\section{Función de Inserción en TimescaleDB (\texttt{insert\_intraday\_metrics})}
\label{annex:code:insert_intraday}
Ejemplo de función en \texttt{db.py} para insertar eficientemente datos en la hipertabla \texttt{intraday\_metrics} utilizando \texttt{psycopg2.extras.execute\_values}. Incluye manejo básico de errores y conflictos.
\begin{lstlisting}[caption={Ejemplo de función de inserción masiva en TimescaleDB (\texttt{db.py}).}, label={lst:insert_intraday_code}]
import psycopg2
from psycopg2.extras import execute_values # Para inserción eficiente
import logging # Mejor usar logging que print para mensajes

# Configurar logging (preferiblemente al inicio de db.py o app.py)
logging.basicConfig(level=logging.INFO, format='%(asctime)s - %(levelname)s - %(message)s')

# Asumiendo que 'conn' es una conexión psycopg2 válida

def insert_intraday_metrics(conn, data_list):
    """
    Inserta una lista de métricas intradía en la hipertabla.
    data_list: lista de tuplas [(time, user_email, type, value), ...]
                time debe ser un objeto datetime con timezone (idealmente UTC).
    """
    if not data_list:
        logging.info("No hay datos intradía para insertar.")
        return True # Nada que insertar

    sql = """
        INSERT INTO intraday_metrics (time, user_email, type, value)
        VALUES %s
        ON CONFLICT (time, user_email, type) DO NOTHING;
        -- Estrategia de conflicto: Ignorar duplicados.
        -- Alternativa: ON CONFLICT (time, user_email, type)
        -- DO UPDATE SET value = EXCLUDED.value; (Actualizar si existe)
    """
    cursor = None # Inicializar cursor fuera del try para el finally
    try:
        cursor = conn.cursor()
        # execute_values es eficiente para inserciones múltiples
        execute_values(cursor, sql, data_list, page_size=100) # page_size ajustable
        conn.commit()
        logging.info(f"Insertadas/Ignoradas {len(data_list)} métricas intradía.")
        return True
    except (Exception, psycopg2.DatabaseError) as error:
        logging.error(f"Error insertando métricas intradía: {error}")
        if conn:
            conn.rollback() # Deshacer transacción en caso de error
        return False
    finally:
        if cursor:
            cursor.close() # Siempre cerrar el cursor

# --- Ejemplo de preparación de datos en fitbit_intraday.py ---
# (Incluir ejemplo de parsing de tiempo y construcción de data_list
#  como en la respuesta anterior, asegurando manejo de timezone)
# from datetime import datetime, timezone
# import pytz
# ... (código de parse_fitbit_time y bucle de procesamiento) ...
\end{lstlisting}
\textit{Nota: La implementación real debe incluir un manejo robusto de zonas horarias, errores de parsing y podría beneficiarse de logging más detallado.}

% --- Sección de consulta ---
\section{Función de Consulta en TimescaleDB (\texttt{get\_hr\_data})}
\label{annex:code:get_hr}
Ejemplo esquemático de una función en \texttt{db.py} para recuperar datos de frecuencia cardíaca para el dashboard, filtrando por usuario y rango de tiempo.
\begin{lstlisting}[caption={Ejemplo de función de consulta de FC (\texttt{db.py}).}, label={lst:get_hr_code}]
import psycopg2
from datetime import datetime
import logging
# import pandas as pd # Opcional: devolver como DataFrame

def get_hr_data(conn, user_email: str, start_date: datetime, end_date: datetime):
    """
    Recupera datos de frecuencia cardíaca para un usuario en un rango de fechas.
    start_date y end_date deben ser objetos datetime (preferiblemente aware, UTC).
    """
    # Asegurarse que las fechas de entrada son conscientes de zona horaria si es necesario
    # O convertir a UTC si la columna 'time' está en UTC
    # Ejemplo: Asumiendo que start/end_date son naive, y BD está en UTC
    # start_date_utc = start_date.replace(tzinfo=timezone.utc)
    # end_date_utc = end_date.replace(tzinfo=timezone.utc)

    sql = """
        SELECT time, value
        FROM intraday_metrics
        WHERE user_email = %s
          AND type = 'heart_rate'
          AND time >= %s -- Fecha/hora de inicio inclusiva
          AND time < %s  -- Fecha/hora de fin exclusiva
        ORDER BY time ASC;
    """
    results = []
    cursor = None
    try:
        cursor = conn.cursor()
        # Pasar las fechas como parámetros
        cursor.execute(sql, (user_email, start_date, end_date))
        results = cursor.fetchall() # Lista de tuplas [(time, value), ...]
        logging.info(f"Recuperados {len(results)} puntos de FC para {user_email}.")

        # --- Opcional: Convertir a DataFrame de Pandas ---
        # if results:
        #     df = pd.DataFrame(results, columns=['time', 'value'])
        #     # Asegurar que la columna de tiempo sea datetime y tenga timezone
        #     df['time'] = pd.to_datetime(df['time'], utc=True)
        #     return df
        # else:
        #     # Devolver DataFrame vacío con columnas definidas
        #     return pd.DataFrame(columns=['time', 'value'])
        # --- Fin Opcional Pandas ---

        return results # Devuelve lista de tuplas por defecto
    except (Exception, psycopg2.DatabaseError) as error:
        logging.error(f"Error recuperando datos de FC para {user_email}: {error}")
        # return pd.DataFrame(columns=['time', 'value']) # O DataFrame vacío
        return [] # Lista vacía en caso de error
    finally:
        if cursor:
            cursor.close()

# --- Ejemplo de uso en el callback del dashboard ---
# db_conn = db.connect_to_db()
# if db_conn:
#     hr_data_tuples = db.get_hr_data(db_conn, selected_email, start_dt_obj, end_dt_obj)
#     db_conn.close()
#     # Procesar hr_data_tuples para generar el gráfico Plotly...
\end{lstlisting}
\textit{Nota: El manejo preciso de las fechas y zonas horarias (\texttt{start\_date}, \texttt{end\_date}) al pasarlas a la consulta SQL es crucial y depende de cómo se almacenen en la BD (con o sin zona horaria) y cómo se reciban del DatePicker.}

% --- Sección de layout de Dash ---
\section{Layout Básico del Dashboard (\texttt{app.py} o \texttt{dashboard.py})}
\label{annex:code:dash_layout}
Estructura básica del layout de la aplicación Dash definida en Python, utilizando los componentes de \texttt{dash} para crear la interfaz interactiva.
\begin{lstlisting}[caption={Ejemplo de layout de Dash integrado en Flask.}, label={lst:dash_layout_code}]
import dash
# En versiones nuevas de Dash (>=2.0):
from dash import dcc, html
# En versiones antiguas:
# import dash_core_components as dcc
# import dash_html_components as html
from dash.dependencies import Input, Output, State
# from flask_login import login_required # Para proteger ruta Flask

# --- Asumiendo 'server' es la instancia de Flask ---
# app_dash = dash.Dash(__name__, server=server, url_base_pathname='/dashboard/')
# # Configurar Dash para servir assets locales si los tienes (CSS, JS)
# # app_dash.config.suppress_callback_exceptions = True # Si callbacks están en otro fichero

# --- Layout de Dash ---
# (Puede estar en app.py o importado de dashboard_layout.py)
# layout = html.Div([ ... ]) # Definición del layout como antes...

def create_dashboard_layout():
    """Función que devuelve el layout para permitir actualizaciones si es necesario."""
    return html.Div([
        html.H1("Panel de Monitorización de Residentes"),
        # Dropdown para seleccionar residente
        html.Div([
            html.Label("Seleccionar Residente:"),
            dcc.Dropdown(
                id='resident-dropdown',
                options=[
                    # Las opciones se cargan dinámicamente con un callback
                ],
                placeholder="Seleccione un residente...",
                clearable=False, # Evitar que quede vacío si solo hay 1 opción
                style={'width': '50%'} # Ajustar estilo si es necesario
            )
        ], style={'padding': 10}),

        # Selector de rango de fechas
        html.Div([
            html.Label("Seleccionar Rango de Fechas:"),
            dcc.DatePickerRange(
                id='date-picker-range',
                start_date_placeholder_text="Fecha Inicio",
                end_date_placeholder_text="Fecha Fin",
                display_format='YYYY-MM-DD',
                # Podrías establecer fechas iniciales por defecto
                # initial_visible_month=datetime.date.today(),
                # start_date=datetime.date.today() - datetime.timedelta(days=7),
                # end_date=datetime.date.today(),
                clearable=True,
                style={'marginLeft': '10px'}
            )
        ], style={'padding': 10, 'display': 'flex', 'alignItems': 'center'}),

        html.Hr(), # Separador visual

        # Contenedor para los gráficos (se actualizan con callbacks)
        html.Div(id='graphs-container', children=[
            html.Div([
                html.H3("Frecuencia Cardíaca"),
                dcc.Loading( # Añadir indicador de carga
                    type="default",
                    children=dcc.Graph(id='hr-graph')
                )
            ]),
            html.Div([
                html.H3("Pasos Diarios"),
                 dcc.Loading(
                    type="default",
                    children=dcc.Graph(id='steps-graph')
                 )
            ]),
            html.Div([
                html.H3("Resumen del Sueño"),
                 dcc.Loading(
                    type="default",
                    children=dcc.Graph(id='sleep-graph')
                 )
            ]),
            # Añadir más gráficos aquí si es necesario
        ])
        # dcc.Store(id='intermediate-data-store') # Para almacenar datos intermedios
    ])

# Asignar el layout a la app Dash
# app_dash.layout = create_dashboard_layout

# --- Ruta Flask para servir el dashboard ---
# @server.route('/dashboard/')
# @login_required # Proteger la ruta
# def dashboard_page():
#     # Renderizar la plantilla base de Flask que contiene el layout de Dash
#     # O directamente servir app_dash.index() si no necesitas plantilla Flask
#     return app_dash.index() # Método estándar para servir Dash standalone/integrado

\end{lstlisting}
\textit{Nota: Se han añadido componentes \texttt{dcc.Loading} para mejorar la experiencia de usuario mientras se cargan los datos.}

% --- Sección de callback de Dash ---
\section{Callback Principal del Dashboard (\texttt{app.py} o \texttt{dashboard\_callbacks.py})}
\label{annex:code:dash_callback}
Ejemplo esquemático del callback principal que actualiza los gráficos del dashboard. Muestra la estructura de inputs, outputs y la lógica de consulta y generación de figuras Plotly.
\begin{lstlisting}[caption={Ejemplo de callback principal en Dash para actualizar gráficos.}, label={lst:dash_callback_code}]
from dash.dependencies import Input, Output, State
import plotly.graph_objs as go
import plotly.express as px # Alternativa para crear figuras más rápido
from datetime import datetime, date, timedelta
import logging
# import db # Asumiendo funciones de db.py
# import pandas as pd # Si se usa Pandas para procesar

# Asumiendo 'app_dash' es la instancia de la app Dash

# --- Callback para poblar el dropdown de residentes (ejecutar al inicio) ---
@app_dash.callback(
    Output('resident-dropdown', 'options'),
    Input('resident-dropdown', 'id') # Input dummy para disparar al cargar
)
def update_resident_options(_):
    options = []
    conn = None
    try:
        conn = db.connect_to_db()
        if conn:
            # Necesitas una función en db.py que devuelva {'name': ..., 'email': ...}
            users = db.get_linked_users(conn)
            options = [{'label': f"{user['name']} ({user['email']})", 'value': user['email']}
                       for user in sorted(users, key=lambda u: u.get('name', ''))] # Ordenar por nombre
        else:
            logging.error("No se pudo conectar a BD para cargar residentes.")
    except Exception as e:
        logging.error(f"Error cargando lista de residentes: {e}")
    finally:
        if conn:
            conn.close()
    return options

# --- Callback principal para actualizar gráficos ---
@app_dash.callback(
    [Output('hr-graph', 'figure'),
     Output('steps-graph', 'figure'),
     Output('sleep-graph', 'figure')],
    [Input('resident-dropdown', 'value'),
     Input('date-picker-range', 'start_date'),
     Input('date-picker-range', 'end_date')],
    # prevent_initial_call=True # Evitar ejecución inicial si no hay valores por defecto
)
def update_graphs(selected_email, start_date_str, end_date_str):
    """
    Callback para actualizar todos los gráficos basado en residente y fechas.
    """
    # --- Crear figuras vacías por defecto ---
    def create_empty_figure(title="Seleccione residente y rango de fechas"):
        fig = go.Figure()
        fig.update_layout(
            title=title,
            xaxis = {"visible": False},
            yaxis = {"visible": False},
            annotations = [{
                "text": "No hay datos para mostrar.",
                "xref": "paper",
                "yref": "paper",
                "showarrow": False,
                "font": {"size": 16}
            }]
        )
        return fig

    hr_fig = create_empty_figure("Frecuencia Cardíaca")
    steps_fig = create_empty_figure("Pasos Diarios")
    sleep_fig = create_empty_figure("Resumen del Sueño")

    # --- Validar Inputs ---
    if not selected_email or not start_date_str or not end_date_str:
        # Si falta algún input, devolver figuras vacías
        return hr_fig, steps_fig, sleep_fig

    # --- Procesar Fechas ---
    try:
        # Convertir string a objeto date (DatePickerRange devuelve date)
        start_date_obj = date.fromisoformat(start_date_str)
        end_date_obj = date.fromisoformat(end_date_str)

        # Convertir a datetime para consultas (inicio del día y fin del día+1)
        # ¡Ajustar según cómo esperen las funciones de BD y la zona horaria!
        start_dt = datetime.combine(start_date_obj, datetime.min.time())
        end_dt = datetime.combine(end_date_obj + timedelta(days=1), datetime.min.time())

    except (ValueError, TypeError) as e:
        logging.error(f"Error procesando fechas: {e}")
        # Devolver figuras vacías si las fechas son inválidas
        return hr_fig, steps_fig, sleep_fig

    # --- Conexión y Consultas a BD ---
    conn = None
    try:
        conn = db.connect_to_db()
        if not conn:
            logging.error("No se pudo conectar a BD para actualizar gráficos.")
            # Actualizar figuras para mostrar error de conexión
            error_title = "Error de Conexión a Base de Datos"
            hr_fig.update_layout(title=error_title)
            steps_fig.update_layout(title=error_title)
            sleep_fig.update_layout(title=error_title)
            return hr_fig, steps_fig, sleep_fig

        # --- Generar Gráfico de Frecuencia Cardíaca ---
        hr_data = db.get_hr_data(conn, selected_email, start_dt, end_dt)
        if hr_data: # Asume lista de tuplas (time, value)
            times, values = zip(*hr_data)
            # Usar Plotly Express para simplificar
            hr_fig = px.line(x=list(times), y=list(values), labels={'x':'Hora', 'y':'Pulsaciones/min'})
            hr_fig.update_layout(title=f'Frecuencia Cardíaca ({selected_email})')
            hr_fig.update_traces(mode='lines+markers') # Añadir marcadores si se desea
        else:
            hr_fig = create_empty_figure(f'Sin datos de FC ({selected_email})')

        # --- Generar Gráfico de Pasos Diarios ---
        steps_data = db.get_steps_data(conn, selected_email, start_date_obj, end_date_obj) # Pasar date
        if steps_data: # Asume lista de tuplas (date, steps)
            dates, steps = zip(*steps_data)
            steps_fig = px.bar(x=list(dates), y=list(steps), labels={'x':'Fecha', 'y':'Número de Pasos'})
            steps_fig.update_layout(title=f'Pasos Diarios ({selected_email})')
        else:
            steps_fig = create_empty_figure(f'Sin datos de Pasos ({selected_email})')

        # --- Generar Gráfico de Sueño ---
        sleep_data = db.get_sleep_data(conn, selected_email, start_dt, end_dt) # Pasar datetime
        if sleep_data: # Asume lista de dicts con fases
            # Procesar para formato apilado (ej. usando Pandas o bucle)
            # df_sleep = pd.DataFrame(sleep_data)
            # df_sleep['date'] = pd.to_datetime(df_sleep['start_time']).dt.date
            # df_melted = df_sleep.melt(id_vars='date',
            #                         value_vars=['minutes_in_rem', 'minutes_in_light',
            #                                     'minutes_in_deep', 'minutes_awake'],
            #                         var_name='Fase', value_name='Minutos')
            # sleep_fig = px.bar(df_melted, x='date', y='Minutos', color='Fase',
            #                    labels={'date':'Noche del', 'Minutos':'Minutos en Fase'})
            # sleep_fig.update_layout(title=f'Fases del Sueño ({selected_email})')

            # Alternativa sin Pandas (más verboso):
            dates_sleep = [log['start_time'].date() for log in sleep_data]
            fig_data = [
                go.Bar(name='REM', x=dates_sleep, y=[log.get('minutes_in_rem', 0) for log in sleep_data]),
                go.Bar(name='Ligero', x=dates_sleep, y=[log.get('minutes_in_light', 0) for log in sleep_data]),
                go.Bar(name='Profundo', x=dates_sleep, y=[log.get('minutes_in_deep', 0) for log in sleep_data]),
                go.Bar(name='Despierto', x=dates_sleep, y=[log.get('minutes_awake', 0) for log in sleep_data])
            ]
            sleep_fig = go.Figure(data=fig_data)
            sleep_fig.update_layout(barmode='stack', title=f'Fases del Sueño ({selected_email})',
                                    xaxis_title='Noche del', yaxis_title='Minutos')

        else:
            sleep_fig = create_empty_figure(f'Sin datos de Sueño ({selected_email})')

    except Exception as e:
        logging.error(f"Error general en callback update_graphs: {e}", exc_info=True)
        # Mostrar error genérico en los gráficos
        error_title_gen = "Error al generar gráficos"
        hr_fig.update_layout(title=error_title_gen)
        steps_fig.update_layout(title=error_title_gen)
        sleep_fig.update_layout(title=error_title_gen)
    finally:
        if conn:
            conn.close() # Asegurarse siempre de cerrar la conexión

    return hr_fig, steps_fig, sleep_fig

\end{lstlisting}
\textit{Nota: Este callback es complejo. Requiere funciones de base de datos robustas, un manejo cuidadoso de los tipos de datos (especialmente fechas/horas) y errores. El uso de Plotly Express puede simplificar la creación de figuras. Se ha añadido logging básico.}

% --- Sección de Implementación de Base de Datos ---
\section{Implementación de Base de Datos}
\label{annex:code:db_implementation}

\subsection{Estructura de Tablas y Consultas}
La implementación utiliza PostgreSQL con la extensión TimescaleDB, organizando los datos en tablas específicas para cada tipo de información temporal:

\begin{lstlisting}[caption={Estructura de tablas principales}, label={lst:table_structure}]
-- Tabla de usuarios
CREATE TABLE users (
    id SERIAL PRIMARY KEY,
    name VARCHAR(255) NOT NULL,
    email VARCHAR(255) NOT NULL,
    access_token TEXT,
    refresh_token TEXT,
    created_at TIMESTAMPTZ DEFAULT CURRENT_TIMESTAMP
);

-- Tabla de métricas intradía
CREATE TABLE intraday_metrics (
    id SERIAL,
    user_id INTEGER REFERENCES users(id),
    time TIMESTAMPTZ NOT NULL,
    type VARCHAR(50) NOT NULL,
    value FLOAT NOT NULL,
    PRIMARY KEY (id)
);

-- Índices optimizados
CREATE INDEX ON intraday_metrics (user_id, time DESC);
CREATE INDEX ON intraday_metrics (time, type);
\end{lstlisting}

\subsection{Implementación de Consultas}
Las consultas están diseñadas para ser eficientes y directas:

\begin{lstlisting}[caption={Ejemplo de consultas implementadas en db.py}, label={lst:query_implementation}]
def get_intraday_metrics(self, user_id, metric_type, start_time=None, end_time=None):
    """Obtiene las métricas intradía de un usuario."""
    query = """
        SELECT time, value FROM intraday_metrics
        WHERE user_id = %s AND type = %s
    """
    params = [user_id, metric_type]
    
    if start_time:
        query += " AND time >= %s"
        params.append(start_time)
    
    if end_time:
        query += " AND time <= %s"
        params.append(end_time)
    
    query += " ORDER BY time"
    
    return self.execute_query(query, params)

def insert_intraday_metric(user_id, timestamp, metric_type, value):
    """Inserta una métrica intradía."""
    query = """
        INSERT INTO intraday_metrics (user_id, time, type, value)
        VALUES (%s, %s, %s, %s)
        ON CONFLICT (user_id, time, type) 
        DO UPDATE SET value = EXCLUDED.value
    """
    return execute_query(query, (user_id, timestamp, metric_type, value))
\end{lstlisting}

\subsection{Gestión de Conexiones}
La clase DatabaseManager implementa una gestión eficiente de conexiones:

\begin{lstlisting}[caption={Implementación del DatabaseManager}, label={lst:db_manager}]
class DatabaseManager:
    def __init__(self):
        self.connection = None
        self.cursor = None

    def connect(self):
        """Establece una conexión a la base de datos."""
        try:
            self.connection = psycopg2.connect(
                host=DB_CONFIG["host"],
                database=DB_CONFIG["database"],
                user=DB_CONFIG["user"],
                password=DB_CONFIG["password"]
            )
            self.cursor = self.connection.cursor()
            return True
        except Exception as e:
            print(f"Error al conectar: {e}")
            return False

    def execute_query(self, query, params=None):
        """Ejecuta una consulta SQL con parámetros opcionales."""
        try:
            self.cursor.execute(query, params)
            if query.strip().upper().startswith('SELECT'):
                return self.cursor.fetchall()
            self.connection.commit()
            return True
        except Exception as e:
            print(f"Error en consulta: {e}")
            self.connection.rollback()
            return None
\end{lstlisting}

Esta implementación proporciona una base sólida para el manejo de datos temporales, con un enfoque en la simplicidad y eficiencia de las consultas. % Si tuvieras más anexos


\end{appendices}

\end{document}
